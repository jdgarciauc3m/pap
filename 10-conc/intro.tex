\section{Introducción}

\subsection{Concurrencia en C++}

\begin{frame}{Motivación}
\begin{itemize}
  \item C++11 ofrece un modelo de concurrencia propio.
  \item Cualquier implementación que cumple el estándar debe proporcionarlo.
    \begin{itemize}
      \item Resuelve problemas inherentes a PThreads.
      \item Portabilidad de código concurrente: Windows, POSIX, ...
    \end{itemize}
  \item Implicaciones:
    \begin{itemize}
      \item Modificaciones en el lenguaje.
      \item Modificaciones en la biblioteca estándar.
    \end{itemize}
  \item Influencia sobre C11 (ISO/IEC 9899:2011).
  \item \alert{Importante}: Concurrencia y paralelismo son conceptos relacionados pero distintos.
\end{itemize}
\end{frame}

\begin{frame}{Estructura}
\begin{itemize}
  \item El lenguaje ofrece:
    \begin{itemize}
      \item Un nuevo modelo de memoria.
      \item Variables \cppkey{thread\_local}.
    \end{itemize}
  \item La biblioteca ofrece:
    \begin{itemize}
      \item Tipos atómicos.
        \begin{itemize}
          \item Útiles para programación libre de cerrojos de forma portable.
        \end{itemize}
      \item Abstracciones portables para la concurrencia.
        \begin{itemize}
          \item \cppid{thread}.
          \item \cppid{mutex}.
          \item \cppid{lock}.
          \item \cppid{packaged\_task}.
          \item \cppid{future}.
        \end{itemize}
    \end{itemize}
\end{itemize}
\end{frame}

\subsection{Hilos}

\begin{frame}[fragile]{Lanzamiento de hilos}
\begin{itemize}
  \item Un hilo representado por la clase \cppid{std::thread}.
    \begin{itemize}
      \item Normalmente representa un hilo del SO.
    \end{itemize}
\begin{lstlisting}
void f1();

struct f2 {
  void operator()();
};

void g() {
  thread t1{f1}; // Lanza un hilo que ejecuta f1()
  thread t2{f2()}; // Lanza un hilo que ejecuta f2::operator()()

  t1.join(); // Espera a que t1 termine.
  t2.join(); // Espera a que t2 termine.
}
\end{lstlisting}
\end{itemize}
\end{frame}

\begin{frame}[fragile]{Objetos compartidos}
\begin{itemize}
  \item Dos hilos pueden acceder a un objeto compartido.
  \item Posibilidad de carreras de datos.
\end{itemize}
\begin{lstlisting}
int x = 42;

void f() { ++x; }
void g() { x=0; }
void h() { cout << "Hola" << endl; }
void i() { cout << "Adios" << endl; }

void carrera() {
  thread t1{f}; thread t2{g};

  t1.join(); t2.join();

  thread t3{h}; thread t4{i};

  t3.join(); t4.join();
}
\end{lstlisting}
\end{frame}

\begin{frame}[fragile]{Paso de argumentos}
\begin{itemize}
  \item Paso de parámetros simplificado sin necesidad de \emph{casts}.
  \item Varias alternativas.
\end{itemize}
\begin{lstlisting}
void f1(int x);

struct f2 {
  void operator()();
  f2(int px) : x{px} {}
  int x;
}

void g() {
  thread t1{f1, 10}; // Ejecuta f1(10)
  thread t2{f2{20}}; // Ejecuta f2{20}.operator()()
  thread t3{[] { f3(42); } }; // Ejecuta una lambda

  t1.join();
  t2.join();
  t3.join();
}
\end{lstlisting}
\end{frame}

\begin{frame}[fragile]{Primera alternativa para resultados}
\begin{itemize}
  \item Devolver resultados usando punteros o referencias.
  \begin{itemize}
    \item Puede tener problemas de condiciones de carrera.
    \item Pero es simple.
  \end{itemize}
\begin{lstlisting}
void suma(const vector<int> & v, int & r) {
  r = accumulate(v.begin(), v.end(), 0)
}

void f() {
  vector<int> w1 {1, 2, 4, 8, 16};
  vector<int> w2 {2, 4, 6, 8, 10};
  int x1, x2;
  thread t1(suma, w1, x1);
  thread t2(suma, w2, x2);
  t1.join();
  t2.join();
  cout << "Suma w1 -> " << x1 << " w2 -> " << x2 << endl;
}
\end{lstlisting}
  \item \alert{Importante}: Correcto mientras no haya datos compartidos.
\end{itemize}
\end{frame}

\subsection{Acceso a datos compartidos}

\begin{frame}[fragile]{Exclusión mutua}
\begin{itemize}
  \item \cppid{mutex} permite controlar el acceso con exclusión mutua a un recurso.
    \begin{itemize}
      \item \cppid{lock()}: Adquiere el cerrojo asociado.
      \item \cppid{unlock()}: Libera el cerrojo asociado.
    \end{itemize}
\begin{lstlisting}
mutex m;
int x = 0;

void f() {
  m.lock();
  ++x;
  m.unlock();
}

void g() {
  thread t1(f); thread t2(f);
  t1.join(); t2.join();

  cout << x << endl;
}
\end{lstlisting}
\end{itemize}
\end{frame}

\begin{frame}[fragile]{Problemas con \texttt{lock()}/\texttt{unlock()}}
\begin{itemize}
  \item Posibles problemas:
    \begin{itemize}
      \item Olvido de liberar cerrojo.
      \item Excepciones.
    \end{itemize}
  \item Solución: \cppid{unique\_lock}.
\begin{lstlisting}
mutex m;
int x = 0;

void f() {
  unique_lock<mutex> l{m}; // Adquiere el cerrojo
  ++x;
} // Libera el cerrojo

void g() {
  thread t1(f); thread t2(f);
  t1.join(); t2.join();

  cout << x << endl;
}
\end{lstlisting}
\end{itemize}
\end{frame}

\begin{frame}[fragile]{Adquisición de múltiples \texttt{mutex}}
\begin{itemize}
  \item La función \cppid{lock()} permite adquirir a la vez varios \cppid{mutex}.
    \begin{itemize}
      \item Adquiere todos o ninguno.
      \item Si alguno está bloqueado espera dejando libres todos.
    \end{itemize}
\begin{lstlisting}
mutex m1, m2, m3;
void f() {
  lock(m1, m2, m3);
  // Acceso a datos compartidos

} // Cuidado: No se liberan los cerrojos
\end{lstlisting}
  \item Especialmente útil en cooperación con \cppid{unique\_lock}
\begin{lstlisting}
void f() {
  unique_lock l1{m1, defer_lock}, unique_lock l2{m2, defer_lock};
  lock(l1, l2);
  // Acceso a datos compartidos

} // Ahora si se liberan los cerrojos
\end{lstlisting}
\end{itemize}
\end{frame}

\subsection{Mecanismos de espera}

\begin{frame}[fragile]{Esperas de tiempo}
\begin{itemize}
  \item Acceso al reloj.
\begin{lstlisting}
using namespace std::chrono;
auto t1 = high_resolution_clock::now();
\end{lstlisting}
  \item Diferencia de tiempos.
\begin{lstlisting}
auto dif = duration_cast<nanoseconds>(t2-t1);
cout << dif.count() << endl;
\end{lstlisting}
  \item Especificación de una espera.
\begin{lstlisting}
this_thread::sleep_for(microseconds{500});
this_thread::sleep_until(t + milliseconds{10});
\end{lstlisting}
\end{itemize}
\end{frame}

\begin{frame}[fragile]{Variables condición}
\begin{itemize}
  \item Mecanismo para sincronizar hilos en acceso a recursos compartidos.
    \begin{itemize}
      \item \cppid{wait()}: Espera en un \cppid{mutex}.
      \item \cppid{notify\_one()}: Despierta a un hilo en espera.
      \item \cppid{notify\_all()}: Despierta a todos los hilos en espera.
    \end{itemize}
\begin{lstlisting}
class peticion;

queue<peticion> cola;
condition_variable cv;
mutex m;

void productor();
void consumidor();
\end{lstlisting}
\end{itemize}
\end{frame}

\begin{frame}[fragile]{Productor}
\begin{lstlisting}
void consumidor() {
  for (;;) {
    unique_lock<mutex> l{m};

    while (cv.wait(l));

    auto m = cola.front();
    cola.pop();
    l.unlock();
   
    procesa(p);
  };
}
\end{lstlisting}
\begin{itemize}
  \item Efecto de \cppid{wait}:
    \begin{enumerate}
      \item Libera el cerrojo y espera una notificación.
      \item Adquiere el cerrojo al despertarse.
    \end{enumerate}
\end{itemize}
\end{frame}

\begin{frame}[fragile]{Consumidor}
\begin{lstlisting}
void productor() {
  for (;;) {
    peticion p = genera();

    unique_lock<mutex> l{m};
    m.push(p);

    cv.notify_one();
  }
}
\end{lstlisting}
\begin{itemize}
\item Efecto de \cppid{notify\_one()}:
  \begin{enumerate}
    \item Despierta a uno de los hilos esperando en la condición.
  \end{enumerate}
\end{itemize}
\end{frame}

\subsection{Futuros y promesas}

\begin{frame}{Devolución de valores}
\begin{itemize}
  \item La combinación de un \cppid{future} y una \cppid{promise} implementan de forma
        eficiente la transferencia de valores entre hilos.
    \begin{itemize}
      \item Evitan la necesidad de hilos.
    \end{itemize}
  \item Idea general
    \begin{itemize}
      \item Cuando un hilo necesita pasar un valor a otro hilo pone el valor en una \emph{promesa}.
      \item La implementación hace que el valor esté disponible en el correspondiente futuro.
    \end{itemize}
  \item Acceso al futuro mediante \cppid{f.get()}:
    \begin{itemize}
      \item Si se ha asignado un valor $\leftarrow$ obtiene el valor.
      \item En otro caso $\leftarrow$ el hilo llamante se bloquea hasta que esté disponible.
    \end{itemize}
  \item Acceso a la promesa mediante \cppid{p.set\_value()}:
    \begin{itemize}
      \item Asigna el valor.
    \end{itemize}
\end{itemize}
\end{frame}

\begin{frame}[fragile]{Futuros y promesas}
\begin{lstlisting}
void f1(promise<int> & p) {
  try {
    int x = calcula();
    p.set_value(x);
  }
  catch (...) {
    p.set_exception(current_exception());
  }
}

void f2(future<int> & f) {
  try {
    int v = f.get();
    procesa(v);
  }
  catch (...) {
    cerr << "Error" << endl;
  }
}
\end{lstlisting}
\end{frame}

\begin{frame}[fragile]{Tareas empaquetadas}
\begin{itemize}
  \item Facilitan la tarea de \emph{conectar} un futuro y una promesa.
    \begin{itemize}
      \item \cppid{packaged\_task} es un envoltorio para poner un valor en una \emph{promesa}.
      \item Permite obtener el \emph{futuro} asociado a la \emph{promesa}: \cppid{get\_future}.
    \end{itemize}
\begin{lstlisting}
int doble(int x) { return x*x; }

void g() {
  packaged_task<int(int)> pt1{doble};
  packaged_task<int(int)> pt2{doble};

  future<int> f1 { pt1.get_future() };
  future<int> f2 { pt2.get_future() };

  thread t1{move(pt1), 10};
  thread t2{move(pt2), 20};

  cout << "La suma es " << f1.get() + f2.get() << endl;
}
\end{lstlisting}
\end{itemize}
\end{frame}

\subsection{Tareas asíncronas}

\begin{frame}{Tareas asíncronas y futuros}
\begin{itemize}
  \item Una tarea asíncrona permite el lanzamiento simple de la ejecución de una tarea:
    \begin{itemize}
      \item En otro hilo de ejecución.
      \item Como una tarea diferida.
    \end{itemize}
  \item Un futuro es un objeto que permite que un hilo pueda devolver un valor a la sección de código que lo invocó.
\end{itemize}
\end{frame}

\begin{frame}[fragile]{Invocación de tareas asíncronas}
\begin{lstlisting}
#include <future>
#include <iostream>

int main() {
  std::future<int> r = std::async(tarea, 1, 10);
  otra_tarea();
  std::cout << "Resultado= " << r.get() << std::endl;
  return 0;
}
\end{lstlisting}
\end{frame}
