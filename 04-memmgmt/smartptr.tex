\section{Punteros inteligentes}

\subsection{Introducción}

\begin{frame}[fragile]{Simplificación la gestión de punteros}
\begin{itemize}
  \item Un patrón de uso típico de uso de punteros es la gestión de memoria dinámica.
\begin{lstlisting}
void f() {
  vector * p = new  vector{1,2,3}};
  g(p); // Podría lanzar excepciones
  if (h(*p)) return;
  delete p;
}
\end{lstlisting}
  \item Problemas:
    \begin{itemize}
      \item Si \cppid{g()} lanza una excepción se genera un goteo de memoria.
      \item Se podría salir de \cppid{g()} sin desasignar la memoria.
      \item Es fácil olvidar algún \cppkey{delete}.
      \item Se puede confundir \cppkey{delete} con \cppkey{delete[]}.
    \end{itemize}
\end{itemize}
\end{frame}

\begin{frame}[t,fragile]{No usar punteros}
\begin{itemize}
  \item Suele ser la solución más simple.
\begin{lstlisting}
void f() {
  vector v{1,2,3};
  g(&v); // Podría lanzar excepciones
  if (h(v)) return;
} // se destruye v
\end{lstlisting}
  \item Recuerda:
    \begin{itemize}
      \item Utiliza la memoria dinámica solamente si es necesario.
    \end{itemize}
\end{itemize}
\end{frame}

\begin{frame}[fragile]{Punteros inteligentes}
\begin{itemize}
  \item Un \alert{puntero inteligente} es un tipo que encapsula un puntero
        y gestiona de forma automática la gestión de la memoria asociada.
    \begin{itemize}
      \item Su destructor libera automáticamente la memoria asociada.
    \end{itemize}
  \item Tipos de punteros inteligentes:
    \begin{itemize}
      \item \cppid{unique\_ptr}: Puntero a un bloque que no admite copias.
      \item \cppid{shared\_ptr}: Puntero con contador de referencias asociado.
      \item \cppid{weak\_ptr}: Puntero auxiliar para \cppid{shared\_ptr}.
      \item \cppid{auto\_ptr}: Despreciado. \alert{No usar}.
    \end{itemize}
\end{itemize}
\end{frame}

\subsection{Punteros compartidos}

\begin{frame}[fragile]{Cuenta de referencias}
\begin{itemize}
  \item Un \cppid{shared\_ptr} mantiene un contador de referencias.
    \begin{itemize}
      \item Cuando se copia se incrementa el contador de referencias.
     \item Cuando se destruye se decrementa el contador y si llega a 0 se desasigna la memoria.
    \end{itemize}
\end{itemize}
\begin{lstlisting}
void f() {
  shared_ptr<string> p1{new string{"Hola"}};
  shared_ptr<string> p2{p1}; // referencias -> 2

  auto n = p1->size(); // string::size(), p1 usado como puntero
  *p1 = "Adios"; // *p1 == *p2 == "Adios"
  if (p2) { // Conversión a bool
    cerr << "Ocupado" << endl;
  }

  p1 = nullptr; // referencias -> 1
  // ...
} // referencias -> 0 ==> Destrucción

\end{lstlisting}
\end{frame}

\begin{frame}[fragile]{Construcción}
\begin{itemize}
  \item Se puede construir por defecto (apunta a \cppkey{nullptr}):
\begin{lstlisting}
shared_ptr<fecha> p1; // p1 == nullptr
\end{lstlisting}
  \item Se pueden copiar punteros:
\begin{lstlisting}
shared_ptr<fecha> p2{p1};
shared_ptr<const fecha> p3{p1};
\end{lstlisting}
  \item Se puede iniciar con un puntero primitivo:
\begin{lstlisting}
shared_ptr<fecha> p3{new fecha};
\end{lstlisting}
  \item Se puede iniciar con objetos temporales de tipo \cppid{unique\_ptr}:
\begin{lstlisting}
unique_ptr<fecha> crea_fecha();
shared_ptr<fecha> p3{crea_fecha()};
\end{lstlisting}
\end{itemize}
\end{frame}

\begin{frame}[fragile]{Acceso a información}
\begin{itemize}
  \item Acceso al puntero primitivo interno.
\begin{lstlisting}
void func_antigua(fecha * pf);

void f() {
  shared_ptr<fecha> p { new fecha };
  func_antigua(p.get());
}
\end{lstlisting}
  \item Acceso a contador de referencias.
\begin{lstlisting}
if (p.unique()) {
  cout << "No hay más referencias" << endl;
}

cout << "Referencias: " << p.use_count() << endl;
\end{lstlisting}
  \item \alert{Aviso}: \cppid{use\_count()} puede ser menos eficiente.
\end{itemize}
\end{frame}

\begin{frame}[fragile]{Eliminación personalizada}
\begin{itemize}
  \item Por defecto, el puntero asociado se libera mediante \cppkey{delete}.
  \item Se puede suministrar otro mecanismo de liberación.
\begin{lstlisting}
conexion * abrir_conexion();
void cerrar_conexion(conexion * pc);

void f() {
  shared_ptr<conexion> p{abrir_conexion(), cerrar_conexion}
  
  // ...

} // cerrar_conexion(p);
\end{lstlisting}
\end{itemize}
\end{frame}

\begin{frame}[t,fragile]{Creación simplificada}
\begin{itemize}
  \item Funciones de creación.
    \begin{itemize}
      \item \cppid{make\_shared} Crea un \cppid{shared\_ptr} usando \cppkey{new}.
    \end{itemize}
\begin{lstlisting}
auto p = make_shared<registro>("Daniel", 43);
\end{lstlisting}
  \item Asigna el objeto y los metadatos con una única operación.
\end{itemize}
\end{frame}

\subsection{Punteros únicos}

\begin{frame}{Punteros únicos}
\begin{itemize}
  \item \cppid{unique\_ptr} ofrece un puntero no compartido que no se puede copiar.
    \begin{itemize}
      \item Soporta semántica de movimiento.
      \item Se puede usar en contenedores y arrays.
      \item Puede apuntar a arrays.
    \end{itemize}
  \item Eficiencia:
    \begin{itemize}
      \item Es más eficiente devolver \cppid{unique\_ptr} que \cppid{shared\_ptr}.
      \item Puede requerir algo más de espacio (un segundo puntero para eliminador).
    \end{itemize}
\end{itemize}
\end{frame}

\begin{frame}{Semántica y usos}
\begin{itemize}
  \item \cppid{unique\_ptr} ofrece semántica de propiedad estricta.
    \begin{itemize}
      \item Posee el objeto apuntado y lo destruye.
      \item No puede ser copiado pero si puede ser movido.
      \item Cuando se destruye, destruye el objeto usando su eliminador (si se suministra uno).
    \end{itemize}
  \item ¿Dónde usar \cppid{unique\_ptr}?
    \begin{itemize}
      \item Seguridad ante excepciones de memoria asignada dinámicamente.
      \item Pasar la propiedad de memoria asignada dinámicamente a una función.
      \item Devolver memoria dinámicamente asignada de una función.
      \item Almacenar punteros a contenedores.
    \end{itemize}
\end{itemize}
\end{frame}

\begin{frame}[fragile]{Seguridad ante excepciones}
\begin{itemize}
  \item Garantizada por la semántica de propiedad estricta.
\begin{lstlisting}
void f(string & s, int x) {
  unique_ptr<int> p { new int{50} };

  string n = s; // Excepción potencial
  if (x < 0) throw out_of_range{};

  *p = 42;
}
\end{lstlisting}
  \item Si se lanza una excepción el destructor de \cppid{unique\_ptr} garantiza
  que se libera la memoria.
\end{itemize}
\end{frame}

\begin{frame}[fragile]{Valor de retorno}
\begin{itemize}
  \item La devolución se basa en la semántica de movimiento.
\begin{lstlisting}
unique_ptr<int> f(int x) {
  unique_ptr<int> p { new int {x} };
  return p;
}

void g() {
  auto z = f(5);
}
\end{lstlisting}
\end{itemize}
\end{frame}


\begin{frame}[fragile]{Paso como parámetro}
\begin{itemize}
  \item Dos alternativas para pasar como parámetro:
    \begin{itemize}
      \item Pasar una referencia.
      \item Usar semántica de movimiento.
    \end{itemize}
\begin{lstlisting}
void f(const unique_ptr<int> & p) {
  *p = 42;
}

unique_ptr<int> g(unique_ptr<int> p) {
  *p = 42;
  return p;
}

void h() {
  unique_ptr<int> q { new int{24} };
  f(q); // OK
  q = g(q); // Error: No se puede copiar
  q = g(move(q)); // OK
}  
\end{lstlisting}
\end{itemize}
\end{frame}

\begin{frame}[fragile]{Dato miembro}
\begin{itemize}
  \item Simplifica la gestión de datos miembros que tradicionalmente
  eran punteros.
\begin{lstlisting}
class X {
public:
  X() : y{} {}
  void pon_valor(Z z) { y.reset(new Y{z}); }
private:
  unique_ptr<Y> y;
};
\end{lstlisting}
  \item Comportamiento por defecto es el deseado
    \begin{itemize}
      \item Destrucción.
      \item Movimiento.
    \end{itemize}
\end{itemize}
\end{frame}

\begin{frame}[fragile]{Uso en contenedores}
\begin{lstlisting}
using mi_vec = vector<unique_ptr<registro>>;

mi_vec genera_registros() {
  mi_vec v;
  for (int i=0; i<max_reg; ++i) {
    unique_ptr<registro> r;
    if (existe(i)) {
      r.reset(new registro{obten_registro(i)});
    }
    v.push_back(r);
  }
  return v;
}
\end{lstlisting}
\end{frame}

\begin{frame}[fragile]{Eliminadores personalizados}
\begin{itemize}
  \item Igual que \cppid{shared\_ptr}.
  \item Ejemplo: Datos obtenidos de una biblioteca que se obtuvieron con
        \cppid{malloc} y tienen que liberarse con \cppid{free}.
\begin{lstlisting}
buffer * get_buffer(int n); // Deben liberarse con free

using free_ptr = void (*)(void *);

void f() {
  unique_ptr<char, free_ptr> buf { get_data(128), free);
  // ...
} // free(buf.get())
\end{lstlisting}
\end{itemize}
\end{frame}

\begin{frame}[fragile]{¿\texttt{make\_unique}?}
\begin{itemize}
  \item La biblioteca estándar no proporciona \cppid{make\_unique} (al estilo de
        \cppid{make\_shared} o \cppid{make\_pair}).
  \item Pero es fácil definirlo:
\end{itemize}
\begin{lstlisting}
template <typename T, typename ... Args>
unique_ptr<T> make_unique(Args && ... args)
{
  return unique_ptr<T>{new T{args...}};
}
\end{lstlisting}
\end{frame}

\begin{frame}[fragile]{Arrays}
\begin{itemize}
  \item \cppid{unique\_ptr} soporta punteros a arrays.
     \begin{itemize}
       \item Es el único puntero inteligente que los soporta.
     \end{itemize}
\begin{lstlisting}
unique_ptr<int[]> crea_lista(int n) {
  unique_ptr<int[]> p{new int[n]};
  for (int i=0; i<n; ++i) {
    p[i] = i;
  }
  return p;
}
\end{lstlisting}
\end{itemize}
\end{frame}

\subsection{Punteros inteligentes y vector}

\mode<presentation>{

\begin{frame}
\begin{block}{vector.h}
\lstinputlisting[lastline=17]{04-memmgmt/vec6/vector.h}
\end{block}
\end{frame}

\begin{frame}
\begin{block}{vector.h}
\lstinputlisting[firstline=19]{04-memmgmt/vec6/vector.h}
\end{block}
\end{frame}

}

\mode<article> {

\begin{frame}
\begin{block}{vector.h}
\lstinputlisting{04-memmgmt/vec6/vector.h}
\end{block}
\end{frame}

}

\mode<presentation> {

\begin{frame}
\begin{block}{vector.cpp}
\lstinputlisting[lastline=15]{04-memmgmt/vec6/vector.cpp}
\end{block}
\end{frame}

\begin{frame}
\begin{block}{vector.cpp}
\lstinputlisting[firstline=17,lastline=30]{04-memmgmt/vec6/vector.cpp}
\end{block}
\end{frame}


\begin{frame}
\begin{block}{vector.cpp}
\lstinputlisting[firstline=32]{04-memmgmt/vec6/vector.cpp}
\end{block}
\end{frame}

}

\mode<article> {

\begin{frame}
\begin{block}{vector.cpp}
\lstinputlisting{04-memmgmt/vec6/vector.cpp}
\end{block}
\end{frame}

}
