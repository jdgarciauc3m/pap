\section{Acceso a elementos}

\subsection{Introducción}

\begin{frame}[fragile]{Operadores de acceso}
\begin{itemize}
  \item Actualmente:
\begin{lstlisting}
v.pon(i,5);
x = v.obten(i);
\end{lstlisting}
  \item \alert{Objetivo}:
\begin{lstlisting}
v[i] = 5;
x = v[i];
\end{lstlisting}
  \item Mecanismo: sobrecarga de operadores.
    \begin{itemize}
      \item Se puede sobrecargar el operador \cppid{[]}.
      \item El operador sobrecargado tiene que ser una función miembro.
      \item Para que la expresión \cppid{v[i]} puede recibir una asignación 
            el operador debe devolve una referencia.
    \end{itemize}
\begin{lstlisting}
class vector {
  // ...
  double & operator[](int i) { return vec[i]; }
  // ...
\end{lstlisting}
\end{itemize}
\end{frame}

\begin{frame}[fragile]{Utilización}
\begin{lstlisting}
void f(vector & v) {
  for (int i=0; i<v.tamanyo(); ++i) {
    v[i] = i  * 2.0;
  }
  cout << v[v.tamanyo()-1] << endl;
}
\end{lstlisting}
\pause
\begin{lstlisting}
void f(const vector & v) {
  for (int i=0; i<v.tamanyo(); ++i) {
    v[i] = i  * 2.0;
  }
  cout << v[v.tamanyo()-1] << endl;
}
\end{lstlisting}
\end{frame}

\begin{frame}[fragile]{Solución}
\begin{itemize}
  \item Sobrecargar el operador \cppid{[]} con dos versiones.
    \begin{itemize}
      \item Una versión \emph{no-constante}.
      \item Una versión \emph{constante}.
    \end{itemize}
\begin{lstlisting}
double & operator[](int i) { return vec[i]; }
double operator[](int i) const { return vec[i]; }
\end{lstlisting}
  \item El sistema de tipos selecciona que llamada realizar.
\end{itemize}
\end{frame}

\subsection{Cambio de tamaño}

\begin{frame}[fragile]{Un vector dinámico}
\begin{itemize}
  \item Objetivo: Permitir que se puedan añadir elementos a un vector.
\begin{lstlisting}
vector v(10);
v.push_back(1.0); // 11 elementos
v.push_back(0.5); // 12 elementos
\end{lstlisting}
  \item Problema: Cuando se crea \cppid{v} tiene asociado un vector de tamaño 10.
  \item Estrategia:
    \begin{itemize}
      \item Diferencia entre
        \begin{itemize}
          \item Tamaño: Número de elementos que tiene el vector.
          \item Capacidad: Número máximo de elementos que caben.
        \end{itemize}
      \item Cuando el vector está lleno reservar otro de mayor tamaño y copiar los elementos.
    \end{itemize}
\end{itemize}
\end{frame}

\begin{frame}{Reserva de espacio}
\begin{itemize}
  \item \cppid{reserva(n)}: Asegura la reserva espacio para \cppid{n} elementos.
    \begin{itemize}
      \item Si \cppid{n} es menor que el espacio actual no hace nada.
      \item Deja los elementos nuevos sin iniciar.
    \end{itemize}
\end{itemize}
\begin{tikzpicture}
\tikzset{
    bloque/.style={rectangle,draw=black, top color=white, bottom color=blue!50,
                   very thick, inner sep=0.5em, minimum size=0.6cm, text centered, font=\tiny},
    bloquelibre/.style={rectangle,draw=black, top color=white, bottom color=blue!10,
                   very thick, inner sep=0.5em, minimum size=0.6cm, text centered, font=\tiny},
    flecha/.style={->, >=latex', shorten >=1pt, thick},
    etiqueta/.style={text centered, font=\tiny} 
}  
\node[bloque] (bsize) {2};
\node[bloque, right=0cm of bsize] (bcap) {4};
\node[bloque,right=0cm of bcap] (bptr) { };
\node[bloque,right=0.5cm of bptr] (v0) {1.0};
\node[bloque,right=0cm of v0] (v1) {2.0};
\node[bloquelibre,right=0cm of v1] (v2) {?};
\node[bloquelibre,right=0cm of v2] (v3) {?};
\draw[flecha] (bptr) -- (v0);
\node[etiqueta, left=0.1cm of bsize] {v1:};
\node[etiqueta, above=0cm of bsize] {tam};
\node[etiqueta, above=0cm of bcap] {cap};
\node[etiqueta, above=0cm of bptr] {vec};
\end{tikzpicture}
\begin{itemize}
  \item \pause \cppid{v.reserva(8)}
\end{itemize}
\begin{tikzpicture}
\tikzset{
    bloque/.style={rectangle,draw=black, top color=white, bottom color=blue!50,
                   very thick, inner sep=0.5em, minimum size=0.6cm, text centered, font=\tiny},
    bloquelibre/.style={rectangle,draw=black, top color=white, bottom color=blue!10,
                   very thick, inner sep=0.5em, minimum size=0.6cm, text centered, font=\tiny},
    flecha/.style={->, >=latex', shorten >=1pt, thick},
    etiqueta/.style={text centered, font=\tiny} 
}  
\node[bloque] (bsize) {2};
\node[bloque, right=0cm of bsize] (bcap) {8};
\node[bloque,right=0cm of bcap] (bptr) { };

\node[bloque,right=0.5cm of bptr] (v0) {1.0};
\node[bloque,right=0cm of v0] (v1) {2.0};
\node[bloquelibre,right=0cm of v1] (v2) {?};
\node[bloquelibre,right=0cm of v2] (v3) {?};
\node[etiqueta, right=0.5cm of v3] (delete) {\cppkey{delete[]}};
\draw[flecha] (delete) -- (v3);

\node[bloque, below right=0.5cm and 0.5cm of bptr] (nv0) {1.0};
\node[bloque,right=0cm of nv0] (nv1) {2.0};
\node[bloquelibre,right=0cm of nv1] (nv2) {?};
\node[bloquelibre,right=0cm of nv2] (nv3) {?};
\node[bloquelibre,right=0cm of nv3] (nv4) {?};
\node[bloquelibre,right=0cm of nv4] (nv5) {?};
\node[bloquelibre,right=0cm of nv5] (nv6) {?};
\node[bloquelibre,right=0cm of nv6] (nv7) {?};

\draw[flecha] (bptr) -- (nv0);
\node[etiqueta, left=0.1cm of bsize] {v1:};
\node[etiqueta, above=0cm of bsize] {tam};
\node[etiqueta, above=0cm of bcap] {cap};
\node[etiqueta, above=0cm of bptr] {vec};
\end{tikzpicture}
\end{frame}

\begin{frame}{Cambio de tamaño}
\begin{itemize}
  \item \cppid{redimiensiona(n)}: Cambia el tamaño de un vector.
    \begin{itemize}
      \item Nuevo tamaño excede capacidad total $\rightarrow$ \cppid{reserva(n)}
      \item Nuevo tamaño mayor que tamaño actual pero no excede capacidad total $\rightarrow$ Iniciar elementos nuevos y actualizar tamaño.
      \item Nuevo tamaño menor que tamaño actual $\rightarrow$ Actualizar tamaño.
    \end{itemize}
\end{itemize}
\pause
\begin{tikzpicture}
\tikzset{
    bloque/.style={rectangle,draw=black, top color=white, bottom color=blue!50,
                   very thick, inner sep=0.5em, minimum size=0.6cm, text centered, font=\tiny},
    bloquelibre/.style={rectangle,draw=black, top color=white, bottom color=blue!10,
                   very thick, inner sep=0.5em, minimum size=0.6cm, text centered, font=\tiny},
    flecha/.style={->, >=latex', shorten >=1pt, thick},
    etiqueta/.style={text centered, font=\tiny} 
}  
\node[bloque] (bsize) {2};
\node[bloque, right=0cm of bsize] (bcap) {4};
\node[bloque,right=0cm of bcap] (bptr) { };
\node[bloque,right=0.5cm of bptr] (v0) {1.0};
\node[bloque,right=0cm of v0] (v1) {2.0};
\node[bloquelibre,right=0cm of v1] (v2) {?};
\node[bloquelibre,right=0cm of v2] (v3) {?};
\draw[flecha] (bptr) -- (v0);
\node[etiqueta, left=0.1cm of bsize] {v1:};
\node[etiqueta, above=0cm of bsize] {tam};
\node[etiqueta, above=0cm of bcap] {cap};
\node[etiqueta, above=0cm of bptr] {vec};
\end{tikzpicture}
\begin{itemize}
  \item \pause \cppid{redimensiona(3)}
\end{itemize}
\pause
\begin{tikzpicture}
\tikzset{
    bloque/.style={rectangle,draw=black, top color=white, bottom color=blue!50,
                   very thick, inner sep=0.5em, minimum size=0.6cm, text centered, font=\tiny},
    bloquelibre/.style={rectangle,draw=black, top color=white, bottom color=blue!10,
                   very thick, inner sep=0.5em, minimum size=0.6cm, text centered, font=\tiny},
    flecha/.style={->, >=latex', shorten >=1pt, thick},
    etiqueta/.style={text centered, font=\tiny} 
}  
\node[bloque] (bsize) {3};
\node[bloque, right=0cm of bsize] (bcap) {4};
\node[bloque,right=0cm of bcap] (bptr) { };
\node[bloque,right=0.5cm of bptr] (v0) {1.0};
\node[bloque,right=0cm of v0] (v1) {2.0};
\node[bloque,right=0cm of v1] (v2) {0.0};
\node[bloquelibre,right=0cm of v2] (v3) {?};
\draw[flecha] (bptr) -- (v0);
\node[etiqueta, left=0.1cm of bsize] {v1:};
\node[etiqueta, above=0cm of bsize] {tam};
\node[etiqueta, above=0cm of bcap] {cap};
\node[etiqueta, above=0cm of bptr] {vec};
\end{tikzpicture}
\end{frame}

\begin{frame}
\begin{tikzpicture}
\tikzset{
    bloque/.style={rectangle,draw=black, top color=white, bottom color=blue!50,
                   very thick, inner sep=0.5em, minimum size=0.6cm, text centered, font=\tiny},
    bloquelibre/.style={rectangle,draw=black, top color=white, bottom color=blue!10,
                   very thick, inner sep=0.5em, minimum size=0.6cm, text centered, font=\tiny},
    flecha/.style={->, >=latex', shorten >=1pt, thick},
    etiqueta/.style={text centered, font=\tiny} 
}  
\node[bloque] (bsize) {3};
\node[bloque, right=0cm of bsize] (bcap) {4};
\node[bloque,right=0cm of bcap] (bptr) { };
\node[bloque,right=0.5cm of bptr] (v0) {1.0};
\node[bloque,right=0cm of v0] (v1) {2.0};
\node[bloque,right=0cm of v1] (v2) {0.0};
\node[bloquelibre,right=0cm of v2] (v3) {?};
\draw[flecha] (bptr) -- (v0);
\node[etiqueta, left=0.1cm of bsize] {v1:};
\node[etiqueta, above=0cm of bsize] {tam};
\node[etiqueta, above=0cm of bcap] {cap};
\node[etiqueta, above=0cm of bptr] {vec};
\end{tikzpicture}
\begin{itemize}
  \item \pause \cppid{redimensiona(1)}
\end{itemize}
\pause
\begin{tikzpicture}
\tikzset{
    bloque/.style={rectangle,draw=black, top color=white, bottom color=blue!50,
                   very thick, inner sep=0.5em, minimum size=0.6cm, text centered, font=\tiny},
    bloquelibre/.style={rectangle,draw=black, top color=white, bottom color=blue!10,
                   very thick, inner sep=0.5em, minimum size=0.6cm, text centered, font=\tiny},
    flecha/.style={->, >=latex', shorten >=1pt, thick},
    etiqueta/.style={text centered, font=\tiny} 
}  
\node[bloque] (bsize) {1};
\node[bloque, right=0cm of bsize] (bcap) {4};
\node[bloque,right=0cm of bcap] (bptr) { };
\node[bloque,right=0.5cm of bptr] (v0) {1.0};
\node[bloquelibre,right=0cm of v0] (v1) {2.0};
\node[bloquelibre,right=0cm of v1] (v2) {0.0};
\node[bloquelibre,right=0cm of v2] (v3) {?};
\draw[flecha] (bptr) -- (v0);
\node[etiqueta, left=0.1cm of bsize] {v1:};
\node[etiqueta, above=0cm of bsize] {tam};
\node[etiqueta, above=0cm of bcap] {cap};
\node[etiqueta, above=0cm of bptr] {vec};
\end{tikzpicture}
\begin{itemize}
  \item \pause \cppid{v.redimensiona(6)}
\end{itemize}
\pause
\begin{tikzpicture}
\tikzset{
    bloque/.style={rectangle,draw=black, top color=white, bottom color=blue!50,
                   very thick, inner sep=0.5em, minimum size=0.6cm, text centered, font=\tiny},
    bloquelibre/.style={rectangle,draw=black, top color=white, bottom color=blue!10,
                   very thick, inner sep=0.5em, minimum size=0.6cm, text centered, font=\tiny},
    flecha/.style={->, >=latex', shorten >=1pt, thick},
    etiqueta/.style={text centered, font=\tiny} 
}  
\node[bloque] (bsize) {6};
\node[bloque, right=0cm of bsize] (bcap) {6};
\node[bloque,right=0cm of bcap] (bptr) { };

\node[bloque,right=0.5cm of bptr] (v0) {1.0};
\node[bloquelibre,right=0cm of v0] (v1) {2.0};
\node[bloquelibre,right=0cm of v1] (v2) {0.0};
\node[bloquelibre,right=0cm of v2] (v3) {?};
\node[etiqueta, right=0.5cm of v3] (delete) {\cppkey{delete[]}};
\draw[flecha] (delete) -- (v3);

\node[bloque,below right=0.5cm and 0.5cm of bptr] (nv0) {1.0};
\node[bloque,right=0cm of nv0] (nv1) {2.0};
\node[bloque,right=0cm of nv1] (nv2) {0.0};
\node[bloque,right=0cm of nv2] (nv3) {0.0};
\node[bloque,right=0cm of nv3] (nv4) {0.0};
\node[bloque,right=0cm of nv4] (nv5) {0.0};

\draw[flecha] (bptr) -- (nv0);

\node[etiqueta, left=0.1cm of bsize] {v1:};
\node[etiqueta, above=0cm of bsize] {tam};
\node[etiqueta, above=0cm of bcap] {cap};
\node[etiqueta, above=0cm of bptr] {vec};
\end{tikzpicture}
\end{frame}

\begin{frame}{Agregación de elementos}
\begin{itemize}
  \item \cppid{agrega\_final(x)}: Agrega un elemento al final del vector modificando su tamaño.
    \begin{itemize}
      \item Si no hay espacio disponible se reserva capacidad adicional.
    \end{itemize}
\end{itemize}
\begin{tikzpicture}
\tikzset{
    bloque/.style={rectangle,draw=black, top color=white, bottom color=blue!50,
                   very thick, inner sep=0.5em, minimum size=0.6cm, text centered, font=\tiny},
    bloquelibre/.style={rectangle,draw=black, top color=white, bottom color=blue!10,
                   very thick, inner sep=0.5em, minimum size=0.6cm, text centered, font=\tiny},
    flecha/.style={->, >=latex', shorten >=1pt, thick},
    etiqueta/.style={text centered, font=\tiny} 
}  
\node[bloque] (bsize) {2};
\node[bloque, right=0cm of bsize] (bcap) {4};
\node[bloque,right=0cm of bcap] (bptr) { };
\node[bloque,right=0.5cm of bptr] (v0) {1.0};
\node[bloque,right=0cm of v0] (v1) {2.0};
\node[bloquelibre,right=0cm of v1] (v2) {?};
\node[bloquelibre,right=0cm of v2] (v3) {?};
\draw[flecha] (bptr) -- (v0);
\node[etiqueta, left=0.1cm of bsize] {v1:};
\node[etiqueta, above=0cm of bsize] {tam};
\node[etiqueta, above=0cm of bcap] {cap};
\node[etiqueta, above=0cm of bptr] {vec};
\end{tikzpicture}
\begin{itemize}
  \item \pause \cppid{agrega\_final(1.5)}
\end{itemize}
\pause
\begin{tikzpicture}
\tikzset{
    bloque/.style={rectangle,draw=black, top color=white, bottom color=blue!50,
                   very thick, inner sep=0.5em, minimum size=0.6cm, text centered, font=\tiny},
    bloquelibre/.style={rectangle,draw=black, top color=white, bottom color=blue!10,
                   very thick, inner sep=0.5em, minimum size=0.6cm, text centered, font=\tiny},
    flecha/.style={->, >=latex', shorten >=1pt, thick},
    etiqueta/.style={text centered, font=\tiny} 
}  
\node[bloque] (bsize) {3};
\node[bloque, right=0cm of bsize] (bcap) {4};
\node[bloque,right=0cm of bcap] (bptr) { };
\node[bloque,right=0.5cm of bptr] (v0) {1.0};
\node[bloque,right=0cm of v0] (v1) {2.0};
\node[bloque,right=0cm of v1] (v2) {1.5};
\node[bloquelibre,right=0cm of v2] (v3) {?};
\draw[flecha] (bptr) -- (v0);
\node[etiqueta, left=0.1cm of bsize] {v1:};
\node[etiqueta, above=0cm of bsize] {tam};
\node[etiqueta, above=0cm of bcap] {cap};
\node[etiqueta, above=0cm of bptr] {vec};
\end{tikzpicture}
\begin{itemize}
  \item \pause \cppid{agrega\_final(2.5)}
\end{itemize}
\pause
\begin{tikzpicture}
\tikzset{
    bloque/.style={rectangle,draw=black, top color=white, bottom color=blue!50,
                   very thick, inner sep=0.5em, minimum size=0.6cm, text centered, font=\tiny},
    bloquelibre/.style={rectangle,draw=black, top color=white, bottom color=blue!10,
                   very thick, inner sep=0.5em, minimum size=0.6cm, text centered, font=\tiny},
    flecha/.style={->, >=latex', shorten >=1pt, thick},
    etiqueta/.style={text centered, font=\tiny} 
}  
\node[bloque] (bsize) {4};
\node[bloque, right=0cm of bsize] (bcap) {4};
\node[bloque,right=0cm of bcap] (bptr) { };
\node[bloque,right=0.5cm of bptr] (v0) {1.0};
\node[bloque,right=0cm of v0] (v1) {2.0};
\node[bloque,right=0cm of v1] (v2) {1.5};
\node[bloque,right=0cm of v2] (v3) {2.5};
\draw[flecha] (bptr) -- (v0);
\node[etiqueta, left=0.1cm of bsize] {v1:};
\node[etiqueta, above=0cm of bsize] {tam};
\node[etiqueta, above=0cm of bcap] {cap};
\node[etiqueta, above=0cm of bptr] {vec};
\end{tikzpicture}
\end{frame}

\begin{frame}
\begin{tikzpicture}
\tikzset{
    bloque/.style={rectangle,draw=black, top color=white, bottom color=blue!50,
                   very thick, inner sep=0.5em, minimum size=0.6cm, text centered, font=\tiny},
    bloquelibre/.style={rectangle,draw=black, top color=white, bottom color=blue!10,
                   very thick, inner sep=0.5em, minimum size=0.6cm, text centered, font=\tiny},
    flecha/.style={->, >=latex', shorten >=1pt, thick},
    etiqueta/.style={text centered, font=\tiny} 
}  
\node[bloque] (bsize) {4};
\node[bloque, right=0cm of bsize] (bcap) {4};
\node[bloque,right=0cm of bcap] (bptr) { };
\node[bloque,right=0.5cm of bptr] (v0) {1.0};
\node[bloque,right=0cm of v0] (v1) {2.0};
\node[bloque,right=0cm of v1] (v2) {1.5};
\node[bloque,right=0cm of v2] (v3) {2.5};
\draw[flecha] (bptr) -- (v0);
\node[etiqueta, left=0.1cm of bsize] {v1:};
\node[etiqueta, above=0cm of bsize] {tam};
\node[etiqueta, above=0cm of bcap] {cap};
\node[etiqueta, above=0cm of bptr] {vec};
\end{tikzpicture}
\begin{itemize}
  \item \pause \cppid{agrega\_final(3.5)}
\end{itemize}
\pause
\begin{tikzpicture}
\tikzset{
    bloque/.style={rectangle,draw=black, top color=white, bottom color=blue!50,
                   very thick, inner sep=0.5em, minimum size=0.6cm, text centered, font=\tiny},
    bloquelibre/.style={rectangle,draw=black, top color=white, bottom color=blue!10,
                   very thick, inner sep=0.5em, minimum size=0.6cm, text centered, font=\tiny},
    flecha/.style={->, >=latex', shorten >=1pt, thick},
    etiqueta/.style={text centered, font=\tiny} 
}  
\node[bloque] (bsize) {5};
\node[bloque, right=0cm of bsize] (bcap) {8};
\node[bloque,right=0cm of bcap] (bptr) { };

\node[bloque,right=0.5cm of bptr] (v0) {1.0};
\node[bloque,right=0cm of v0] (v1) {2.0};
\node[bloque,right=0cm of v1] (v2) {1.5};
\node[bloque,right=0cm of v2] (v3) {2.5};
\node[etiqueta, right=0.5cm of v3] (delete) {\cppkey{delete[]}};
\draw[flecha] (delete) -- (v3);

\node[bloque,below right=0.5cm and 0.5cm of bptr] (nv0) {1.0};
\node[bloque,right=0cm of nv0] (nv1) {2.0};
\node[bloque,right=0cm of nv1] (nv2) {1.5};
\node[bloque,right=0cm of nv2] (nv3) {2.5};
\node[bloque,right=0cm of nv3] (nv4) {3.5};
\node[bloquelibre,right=0cm of nv4] (nv5) {?};
\node[bloquelibre,right=0cm of nv5] (nv6) {?};
\node[bloquelibre,right=0cm of nv6] (nv7) {?};

\draw[flecha] (bptr) -- (nv0);

\node[etiqueta, left=0.1cm of bsize] {v1:};
\node[etiqueta, above=0cm of bsize] {tam};
\node[etiqueta, above=0cm of bcap] {cap};
\node[etiqueta, above=0cm of bptr] {vec};
\end{tikzpicture}
\end{frame}

\subsection{Un vector evolucionado}

\mode<presentation>{

\begin{frame}
\begin{block}{vector.h}
\lstinputlisting[lastline=20]{04-memmgmt/vec7/vector.h}
\end{block}
\end{frame}

\begin{frame}
\begin{block}{vector.h}
\lstinputlisting[firstline=22]{04-memmgmt/vec7/vector.h}
\end{block}
\end{frame}

}

\mode<article>{

\begin{frame}
\begin{block}{vector.h}
\lstinputlisting{04-memmgmt/vec7/vector.h}
\end{block}
\end{frame}

}

\mode<presentation>{

\begin{frame}
\begin{block}{vector.cpp}
\lstinputlisting[lastline=17]{04-memmgmt/vec7/vector.cpp}
\end{block}
\end{frame}

\begin{frame}
\begin{block}{vector.cpp}
\lstinputlisting[firstline=19,lastline=34]{04-memmgmt/vec7/vector.cpp}
\end{block}
\end{frame}

\begin{frame}
\begin{block}{vector.cpp}
\lstinputlisting[firstline=36,lastline=49]{04-memmgmt/vec7/vector.cpp}
\end{block}
\end{frame}

\begin{frame}
\begin{block}{vector.cpp}
\lstinputlisting[firstline=51,lastline=62]{04-memmgmt/vec7/vector.cpp}
\end{block}
\end{frame}

\begin{frame}
\begin{block}{vector.cpp}
\lstinputlisting[firstline=64]{04-memmgmt/vec7/vector.cpp}
\end{block}
\end{frame}

}

\mode<article>{
\begin{frame}
\begin{block}{vector.cpp}
\lstinputlisting[lastline=15]{04-memmgmt/vec7/vector.cpp}
\end{block}
\end{frame}
}

\begin{frame}[fragile]
\begin{block}{main.cpp}
\mode<presentation>{
\lstinputlisting[basicstyle=\tiny\ttfamily]{04-memmgmt/vec7/main.cpp}
}
\mode<article>{
\lstinputlisting{04-memmgmt/vec7/main.cpp}
}
\end{block}
\begin{block}{Salida}
\begin{lstlisting}[style=terminal]
v1: (1 2 3 4 )
v2: (2.5 3.5 0 0 )
v3: ()
\end{lstlisting}
\end{block}
\end{frame}

