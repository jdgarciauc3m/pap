\section{Introducción}

\begin{frame}[t]{¿Qué es un contenedor?}
\begin{itemize}
  \item Un \textgood{contenedor} es una estructura de datos capaz de almacenar objetos.
  \vfill\pause
  \item Tipos de contenedores:
    \begin{itemize}
      \item \textmark{Contenedores de secuencia}: Almacenan una secuencia de valores.
      \item \textmark{Contenedores asociativos}: Almacenan pares \emph{clave}-\emph{valor}.
    \end{itemize}
  \vfill\pause
  \item Tipos que son parcialmente contenedores:
    \begin{itemize}
      \item \textmark{Adaptadores de contenedor}: Acceso especializado a otro contenedor.
      \item \textmark{Casi contenedores}: Ofrecen solo algunas de las propiedades de un contenedor.
    \end{itemize}
\end{itemize}
\end{frame}

\begin{frame}[t,fragile]{Contenedores y gestión de memoria}
\begin{itemize}
  \item Todos los contenedores de la biblioteca estándar toman como argumento de plantilla
        un asignador de memoria (\emph{allocator}).
    \begin{itemize}
      \item Es un tipo \cppid{A} que abstrae la forma de gestionar la memoria.
      \item Existe una clase por con un asignador básico (\cppid{std::allocator<T>}).
      \item El tipo \cppid{std::alocator<T>} es el valor por defecto si no se pasa otro.
    \end{itemize}
\end{itemize}
\vfill
\begin{lstlisting}
template <typename T, typename A=allocator<T>>
class vector {
  // ...
};
\end{lstlisting}
\end{frame}

\begin{frame}[t]{Contenedores de secuencia}
\begin{itemize}
  \item Almacenan una lista de valores de un determinado tipo.
  \item Permiten acceder a los valores como una secuencia semi-abierta.
  \vfill\pause
  \item Tipos:
    \begin{itemize}
      \item \cppid{vector<T,A>}: Secuencia asignada de forma contigua.
      \item \cppid{list<T,A>}: Lista de nodos doblemente enlazada.
      \item \cppid{forward\_list<T,A>}: Lista de elementos simplemente enlazada.
      \item \cppid{deque<T,A>}: Cola de doble extremo (\emph{deque}).
    \end{itemize}
\end{itemize}
\end{frame}

\begin{frame}[t]{Contenedores asociativos}
\begin{itemize}
  \item Almacenando una colección de pares \emph{clave-valor} permitiendo buscar por clave.
  \vfill\pause
  \item Dos categorías:
    \begin{itemize}
      \item \textmark{Ordenados}: Las claves están sujetas a un criterio de ordenación.
        \begin{itemize}
          \item Típicamente implementadas mediante algún tipo de árbol.
        \end{itemize}
      \item \textmark{No-ordenados}: Las claves permiten la aplicación de una función \emph{hash}
            y son comparables en igualdad.
        \begin{itemize}
          \item Típicamente implementadas mediante algún tipo de tabla \emph{hash}.
        \end{itemize}
    \end{itemize}
  \vfill\pause
  \item Alternativas en cuanto a claves:
    \begin{itemize}
      \item \textmark{Claves duplicadas}: Se pueden tener varios valores con la misma clave.
      \item \textmark{Claves únicas}: No se puede tener más de un valor con la misma clave.
    \end{itemize}
\end{itemize}
\end{frame}

\begin{frame}[t]{Tipos para contenedores asociativos}
\vspace{-1em}
\begin{itemize}
  \item Tipos \textgood{ordenados}:
    \begin{itemize}
      \item \cppid{map<K,V,C,A>}: Secuencia de pares \cppid{<K,V>} sin claves duplicadas.
      \item \cppid{multimap<K,V,C,A>}: Secuencia de pares \cppid{<K,V>} con claves duplicadas.
      \item \cppid{set<K,C,A>}: Conjunto de valores \cppid{<K>} sin duplicados.
      \item \cppid{multi\_set<K,C,A>}: Conjunto de valores \cppid{<K>} con duplicados.
    \end{itemize}
  \vfill\pause
  \item Tipos \textgood{no ordenados}:
    \begin{itemize}
      \item \cppid{unordered\_map<K,V,C,A>}: Colección de pares \cppid{<K,V>} sin claves duplicadas.
      \item \cppid{unordered\_multimap<K,V,C,A>}: Colección de pares \cppid{<K,V>} con claves duplicadas.
      \item \cppid{unordered\_set<K,C,A>}: Conjunto de valores \cppid{<K>} sin duplicados.
      \item \cppid{unordered\_multiset<K,C,A>}: Conjunto de valores \cppid{<K>} con duplicados.
    \end{itemize}
\end{itemize}
\end{frame}

\begin{frame}[t]{Adaptadores de contenedores}
\begin{itemize}
  \item Suministran una interfaz especializada a partir de otro contenedor.
  \item Tipos:
    \begin{itemize}
      \item \cppid{stack<T,C>}: Ofrece una interfaz de pila para un contenedor \cppid{C} y 
            valores de tipo \cppid{T}.
      \item \cppid{queue<T,C>}: Ofrece una interfaz de cola para un contenedor \cppid{C} y 
            valores de tipo \cppid{T}.
      \item \cppid{priority\_queue<T,C,Cmp>}: Ofrece una interfaz de cola con prioridades 
            para un contenedor \cppid{C} y valores de tipo \cppid{T}. El criterio \cppid{Cmp}
            se usa para ordenar por prioridades.
    \end{itemize}
\end{itemize}
\end{frame}

\begin{frame}[t]{Casi contenedores}
\begin{itemize}
  \item Son tipos que la biblioteca estándar puede tratar en muchos contextos como un
        contenedor.
  \item Tipos de \emph{array}:
    \begin{itemize}
      \item \cppid{T[N]}: Array primitivo de tamaño \cppid{N}.
        \begin{itemize}
          \item No tiene funciones miembro (\cppid{begin()}, \cppid{end()}, size(), \ldots).
        \end{itemize}
      \item \cppid{array<T,N>}: Evoltorio para array con algunas funciones.
      \item \cppid{valarray<T>}: Vector numérico posiblemente con operaciones vectorizadas.
    \end{itemize}
  \item Tipos de cadena:
    \begin{itemize}
      \item \cppid{basic\_string<C,T,A>}: Tipo cadena para cualquier tipo carácter \cppid{C}.
      \item \cppid{string}, \cppid{u16string}, \cppid{u32string}, \cppid{wstring}.
    \end{itemize}
  \item Manipulación de bits:
    \begin{itemize}
      \item \cppid{bitset<N>}: Conjunto de \cppid{N} bits.
      \item \cppid{vector<bool>}: Especialización de \cppid{vector} con representación compacta.
    \end{itemize}
\end{itemize}
\end{frame}

\begin{frame}[t,fragile]{Tipos miembro}
\begin{itemize}
  \item Todos los contenedores contienen una serie de definiciones de sinómimos de tipo.
    \begin{itemize}
      \item Simplifican la programación genérica.
    \end{itemize}
\end{itemize}
\begin{block}{Ejemplo}
\begin{lstlisting}
//...
vector<contrapartida>::size_type n = obten_tam();
vector<contrapartida> res(n);
rellena(res);
for (vector<contrapartida>::iterator i= res.begin(); i!=res.end();++i) {
  i->normaliza_valores();
}
//...
\end{lstlisting}
\end{block}
\pause
\begin{itemize}
  \item Muchos usos hoy día innecesarios gracias a \cppkey{auto}.
\end{itemize}
\end{frame}

\begin{frame}[t]{Tipos miembro para todos}
\begin{itemize}
  \item Sinónimos para tipos del valor:
  \begin{itemize}
    \item \cppid{value\_type}: Tipo de valor almacenado.
    \item \cppid{reference}: Sinónimo \cppid{value\_type\&}.
    \item \cppid{const\_reference}: Sinónimo \cppid{const value\_type\&}.
    \item \cppid{pointer}: Sinónimo \cppid{value\_type*}.
    \item \cppid{const\_pointer}: Sinónimo \cppid{const value\_type*}.
  \end{itemize}
  \item Tipos para asignador y tamaño:
  \begin{itemize}
    \item \cppid{allocator\_type}: Tipo del asignador usado.
    \item \cppid{size\_type}: Tipo para representar el tamaño.
  \end{itemize}
\end{itemize}
\end{frame}

\begin{frame}[t]{Tipos miembro para todos (iteradores)}
\begin{itemize}
  \item Sinónimos para tipos que permiten recorrer contenedores.
  \begin{itemize}
    \item \cppid{iterator}: Tipo iterador que se comporta como \cppid{value\_type*}.
    \item \cppid{const\_iterator}: Tipo iterador que se comporta como \cppid{const value\_type*}.
    \item \cppid{reverse\_iterator}: Tipo iterador que se comporta como \cppid{value\_type*}.
    \item \cppid{const\_reverse\_iterator}: Tipo iterador que se comporta como \cppid{const value\_type*}.
  \end{itemize}
\end{itemize}
\end{frame}

\begin{frame}[t]{Tipos miembro para contenedores asociativos}
\begin{itemize}
  \item Sinónimos para todos los contenedores asociativos:
    \begin{itemize}
      \item \cppid{key\_type}: Tipo para la clave.
      \item \cppid{mapped\_type}: Tipo para el valor.
      \item \cppid{key\_compare}: Tipo para el criterio de comparación.
    \end{itemize}
  \vfill\pause
  \item Sinónimos solamente para los contenedores asociativos no ordenados:
    \begin{itemize}
      \item \cppid{hasher}: Tipo para la función \emph{hash}.
      \item \cppid{key\_equal}: Tipo para el criterio de comparación.
      \item \cppid{local\_iterator}: Tipo para el iterador dentro de una categoría.
      \item \cppid{const\_local\_iterator}: Tipo para el iterador constante dentro de una categoría.
    \end{itemize}
\end{itemize}
\end{frame}
