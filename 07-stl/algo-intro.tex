\section{Introducción a algoritmos}

\begin{frame}[t,fragile]{Secuencias}
\begin{itemize}
  \item Algoritmos sobre secuencias:
    \begin{itemize}
      \item En vez de operar sobre contenedores los algoritmos operan sobre secuencias.
    \end{itemize}
  \item Concepto de secuencia:
    \begin{itemize}
      \item Una especificación de una sucesión de valores de un contenedor.
      \item Especificadas mediante un par de iteradores como rango semiabierto \cppid{[inicio,fin)}.
    \end{itemize}
\end{itemize}
\begin{lstlisting}
sort(c1.begin(), c1.end()); 
sort(c2.begin(), c2.end());
sort(c3.begin(), c3.begin()+5);
\end{lstlisting}
\end{frame}

\begin{frame}[t,fragile]{Uso de políticas}
\begin{itemize}
  \item Algunos algoritms toman una política como argumento adicional.
\end{itemize}
\begin{lstlisting}
template <typename I>
void sort(I first, I last);

template <typename I, typename C>
void sort(I first, I last, C cmp);

vector<int> v = obten_vector();
sort(v.begin(), v.end()); // Ordena con <
sort(v.begin(), v.end(), 
  [](int x, int y) { return abs(x) < abs(y); });
\end{lstlisting}
\end{frame}

\begin{frame}[t]{Clasificación de algoritmos}
\begin{itemize}
  \item Algoritmos de secuencia no modificantes.
    \begin{itemize}
      \item Usan como entrada los valores de una secuencia.
    \end{itemize}
  \item Algoritmos de secuencia modificantes.
    \begin{itemize}
      \item Modifican los valores de una secuencia.
    \end{itemize}
  \item Algoritmos de ordenación y búsqueda.
    \begin{itemize}
      \item Reorganizan una secuencia o buscan sobre ella.
    \end{itemize}
  \item Algoritmos de mínimo y máximo.
    \begin{itemize}
      \item Obtienen el mínimo el máximo o ambos.
    \end{itemize}
  \item Algoritmos numéricos.
    \begin{itemize}
      \item Realizan cálculos sobre una secuencia de números.
    \end{itemize}
\end{itemize}
\end{frame}
