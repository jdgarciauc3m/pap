\section{Algoritmos modificantes}

\begin{frame}[t,fragile]{Transformaciones}
\begin{itemize}
  \item \cppid{it = transform(i,f,o,fun)}: Aplica la función \cppid{fun(x)} a cada elemento
        de la secuencia \cppid{[i,f)} 
        y almacena el resultado en \cppid{[o,o+(f-i))}.
  \item \cppid{it = transform(i,f,i2,o,fun)}: Aplica la función \cppid{fun(x,y)} a cada elemento
        de las secuencias \cppid{[i,f)} y \cppid{[i2,i2+(f-i))} 
        y almacena el resultado en \cppid{[o,o+(f-i))}.
\end{itemize}
\begin{lstlisting}
void toupper(string & s) {
  transform(s.begin(), s.end(), s.begin(), toupper); 
}
\end{lstlisting}
\end{frame}

\begin{frame}[t,fragile]{Ejemplo: Suma de vectores}
\begin{lstlisting}
vector<double> sumavec(const vector<double> & x, const vector<double> & y) {
  vector<double> r;
  if (x.size() != y.size()) return r;
  r.resize(x.size());
  transform(x.begin(), x.end(), y.begin(), r.begin(),
    [](double a, double b) { return a+b; }
  );
  return r;
}
\end{lstlisting}
\end{frame}

\begin{frame}[t]{Copia}
\begin{itemize}
  \item \cppid{it = copy(i,f,o)}: Copia todos los elementos de \cppid{[i,f)} 
        a partir de \cppid{o}.
  \item \cppid{it = copy\_if(i,f,o,pred)}: Copia todos los elementos de \cppid{[i,f)} 
        que cumplan el predicado \cppid{pred(x)} 
        a partir de \cppid{o}.
  \item \cppid{it = copy\_n(i,f,o)}: Copia los primeros \cppid{n} elementos de \cppid{[i,i+n)} 
        a partir de \cppid{o}.
  \item \cppid{it = copy\_backward(i,f,o)}: Copia todos los elementos de \cppid{[i,f)} 
        a partir de \cppid{o},
        realizando las copias de fin a principio.
\end{itemize}
\end{frame}

\begin{frame}[t]{Movimiento}
\begin{itemize}
  \item \cppid{it = move(i,f,o)}: Mueve todos los elementos de \cppid{[i,f)} 
        a partir de \cppid{o}.
  \item \cppid{it = move\_backward(i,f,o)}: Mueve todos los elementos de \cppid{[i,f)} 
        a partir de \cppid{o},
        realizando las copias de fin a principio.
\end{itemize}
\end{frame}

\begin{frame}[t]{Unique}
\begin{itemize}
  \item \cppid{it = unique(i,f)}: Reorganiza los elementos de forma que se mueven 
        todos los duplicados al final.
  \item \cppid{it = unique(i,f,pred)}:  Reorganiza los elementos de forma que se mueven 
        todos los duplicados (según el criterio \cppid{pred(x,y)}) al final.
  \item \cppid{it = unique\_copy(i,f,o)}: Copia los elementos de \cppid{[i,f)}
        a partir de \cppid{o} saltando los duplicados.
  \item \cppid{it = unique\_copy(i,f,o,pred)}: Copia los elementos de \cppid{[i,f)}
        a partir de \cppid{o} saltando los duplicados y usando \cppid{pred(x,y)} como criterio.
\end{itemize}
\end{frame}

\begin{frame}[t]{Eliminación}
\begin{itemize}
  \item \cppid{it = remove(i,f,v)}: Elimina los elementos con el valor \cppid{v} de la secuencia \cppid{[i,f)}.
  \item \cppid{it = remove(i,f,pred)}: Elimina los elementos que cumplen \cppid{pred(x)} de la secuencia \cppid{[i,f)}.
  \item \cppid{it = remove\_copy(i,f,o,v)}: Copia los elementos distintos de \cppid{v} desde la secuencia \cppid{[i,f)}
        a la secuencia \cppid{[o,o+(f-i))}.
  \item \cppid{it = remove\_copy\_if(i,f,o,pred)}: Copia los elementos que no cumplen \cppid{pred(x)} desde la secuencia \cppid{[i,f)}
        a la secuencia \cppid{[o,o+(f-i))}.
  \vfill
  \item \alert{NOTA}: \emph{Eliminar} significa moverlos al final.
\end{itemize}
\end{frame}

\begin{frame}[t]{Inversión}
\begin{itemize}
  \item \cppid{reverse(i,f)}: Invierte el orden de la secuencia \cppid{[i,f)}.
  \item \cppid{reverse\_copy(i,f,o)}: Copia la secuencia \cppid{[i,f)} en orden inverso a partir de \cppid{o}.
\end{itemize}
\end{frame}

\begin{frame}[t]{Remplazo}
\begin{itemize}
  \item \cppid{replace(i,f,v1,v2)}: Sustituye todos los elementos con el valor \cppid{v1}
        en la secuencia \cppid{[i,f)}
        por el valor\cppid{v2}.
  \item \cppid{replace(i,f,pred,v2)}: Sustituye todos los elementos que cumplen el predicado \cppid{pred(x)}
        en la secuencia \cppid{[i,f)}
        por el valor\cppid{v2}.
  \item \cppid{it = replace\_copy(i,f,o,v1,v2)}: Copia la secuencia \cppid{[i,f)} a partir de \cppid{o}
        sustituyendo las ocurrencias de \cppid{v1} por \cppid{v2}.
  \item \cppid{it = replace\_copy(i,f,o,pred,v2)}: Copia la secuencia \cppid{[i,f)} a partir de \cppid{o}
        sustituyendo las ocurrencias que cumplen \cppid{pred(x)} por \cppid{v2}.
\end{itemize}
\end{frame}

\begin{frame}[t]{Rotación, mezcla y partición}
\begin{itemize}
  \item Se puede realizar una rotación de los elementos de una secuencia (\cppid{rotate}, \cppid{rotate\_copy}).
  \item Se pueden mezclar los valores de forma aleatoria (\cppid{random\_shuffle}, \cppid{shuffle}).
  \item Se puede realizar una \emph{partición} de un conjunto de valores (\cppid{partition}, 
        \cppid{stable\_partition}, \cppid{partition\_copy}, \cppid{partition\_point}, \cppid{is\_partition}).
  \item Se pueden generar permutaciones (\cppid{next\_permutation}, \cppid{prev\_permutation}, \cppid{is\_permutation}).
\end{itemize}
\end{frame}

\begin{frame}[t]{Llenado}
\begin{itemize}
  \item \cppid{fill(i,f,v)}: Asigna \cppid{v} a cada elemento de \cppid{[i,f)}.
  \item \cppid{it = fill\_n(i,n,v)}: Asigna \cppid{v} a cada elemento de \cppid{[i,i+n)}.
  \item \cppid{generate(i,f,g)} Asigna el resultado de \cppid{g()} a cada elemento de \cppid{[i,f)}.
  \item \cppid{it = generate(i,n,g)} Asigna el resultado de \cppid{g()} a cada elemento de \cppid{[i,i+n)}.
\end{itemize}
\end{frame}

\begin{frame}[t]{Intercambio}
\begin{itemize}
  \item \cppid{swap(x,y)}: Intercambia \cppid{x} e \cppid{y} (de cualquier tipo).
  \item \cppid{it = swap\_ranges(i,f,i2)}: Intercambia los valores de \cppid{[i,f)} y \cppid{[i2, i2+(f-i))}.
  \item \cppid{iter\_swap(i,j)}: Intercambia \cppid{*i} y \cppid{*j}
\end{itemize}
\end{frame}
