\section{Iteradores}

\begin{frame}[t]{¿Qué es un iterador?}
\begin{itemize}
  \item Es un valor que permite recorrer las posiciones de un contendor.
  
  \vfill
  \item Operaciones mínimas:
    \begin{itemize}
      \item Acceder al objeto apuntado (\cppid{*i}).
      \item Avanzar al siguiente elemento (\cppid{++i}).
      \item Comprar dos iteradores (\cppid{i!=j}).
    \end{itemize}

  \vfill
  \item Posiciones especiales:
    \begin{itemize}
      \item \textmark{Inicio}: Apunta al primer elemento.
      \item \textmark{Fin}: Apunta a uno despúes del último elemento.
    \end{itemize}
\end{itemize}
\end{frame}

\begin{frame}[t,fragile]{Iteradores a partir de contenedor}
\begin{itemize}
  \item \cppid{begin()} y \cppid{end()}: Inicio y fin de contenedor.
  \item \cppid{cbegin()} y \cppid{cend()}: Inicio y fin constantes.
  \item \cppid{rbegin()} y \cppid{rend()}: Inicio y fin en orden inverso.
  \item \cppid{crbegin()} y \cppid{crend()}: Inicio y fin constantes en orden inverso.
\end{itemize}
\vfill\pause
\begin{lstlisting}
for (vector<int>::iterator i=v.begin(); i!=v.end(); ++i) {
  cout << *i << endl;
}

for (auto j=v.begin(); j!=v.end(); ++j) {
  cout << *j << endl;
}

for (auto & x : v) {
  cout << x << endl;
}
\end{lstlisting}
\end{frame}
