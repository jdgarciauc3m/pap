\section{Acceso a elementos}

\begin{frame}[t,fragile]{Acceso en contenedores de secuencia}
\begin{itemize}
  \item Acceso a los extremos:
    \begin{itemize}
      \item \cppid{c.front()}: Primer elemento.
      \item \cppid{c.back()}: Último elemento (excepto \cppid{forward\_list}).
    \end{itemize}

  \vfill\pause
  \item Acceso indexado (excepto listas):
    \begin{itemize}
      \item \cppid{c[n]}: Acceso al elemento n-ésimo. Sin verificación.
      \item \cppid{c.at(n)}: Acceso al elemento n-ésimo. Con verificación.
    \end{itemize}
\end{itemize}
\begin{lstlisting}
vector<double> v{1.0, 1.5, 2.5, 4.0};
double a = v.front(); // a==1.0
double b = v.back();  // a==4.0
v[2]=33.3; {1.0, 1.5, 33.3, 4.0}
double c = v[0]; //c==1.0

c[8]=1.0; // No definido
c.at(8); // throw out_of_range{}
\end{lstlisting}
\end{frame}

\begin{frame}[t,fragile]{Acceso en contenedores asociativos}
\begin{itemize}
  \item Acceso indexado por clave (solamente \cppid{map} y \cppid{unordered\_map}).
    \begin{itemize}
      \item \cppid{c[k]}: Acceso al elemento con clave \cppid{k}.
        \begin{itemize}
          \item Si no encuentra \cppid{k}, inserta el par \cppid{<k,V{}>}.
        \end{itemize}
      \item \cppid{c.at(k)}: Acceso al elemento con clave \cppid{k}.
        \begin{itemize}
          \item Si no encuentra \cppid{k}, lanza \cppid{out\_of\_range}.
        \end{itemize}
    \end{itemize}
\end{itemize}
\begin{lstlisting}
map<string,double> nota;
nota["carlos"]=9;
cout << nota.at("carlos") << endl;
\end{lstlisting}
\end{frame}
