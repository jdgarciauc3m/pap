\section{Tamaño y capacidad}

\begin{frame}[t,fragile]{Tamaño del contenedor}
\begin{itemize}
  \item Todos los cotenedores tienen asociado:
    \begin{itemize}
      \item Un \textmark{tamaño} o número actual de elementos.
      \item Un \textmark{tamaño máximo} o número máximo de elementos.
    \end{itemize}
\end{itemize}
\vfill
\begin{lstlisting}
vector<int> v{1,2,3,4,5};
n = v.size(); // n==5

if (v.empty()) { cout << "vacio" << endl; }

cout << "max: " << v.max_size() << endl;

v.clear(); // Erase all elements. size==0
\end{lstlisting}
\end{frame}

\begin{frame}[t,fragile]{Capacidad del contenedor}
\begin{itemize}
  \item Algunos contenedores (\cppid{vector}, \cppid{string}) tienen asociada una \textmark{capacidad}:
    \begin{itemize}
      \item Número máximo de elementos sin necesidad de nuevas asignaciones de memoria.
    \end{itemize}
\end{itemize}
\begin{lstlisting}
vector<int> v{1,2,3,4,5};
v = v.capacity(); // Capacidad actual

v.reserve(50); // v tiene capacidad para hasta 50 elem.

v.resize(10); // Cambia tamaño a 10, añadiendo valor por defecto
v.resize(15,42); // Cambia a tamaño, añadiendo valor 42

v.shrint_to_fit(); // v tiene capacidad igual v.size()
\end{lstlisting}
\end{frame}
