\section{Constructores}

\subsection{Tipos sin constructor}

\begin{frame}[fragile]{Iniciación sin constructor}
\begin{itemize}
  \item Iniciación de tipos primitivos.
\begin{lstlisting}
int x{1024}; // int x = 1024
double * p{nullptr}; // double * p = nullptr
\end{lstlisting}
  \item Clases que no tienen constructor:
\begin{lstlisting}
struct dispositivo {
  string id_serie;
  long long capacidad;
};
\end{lstlisting}
    \begin{itemize}
      \item Iniciación por miembros:
\begin{lstlisting}
dispositivo d1{"hda1", 1024};
\end{lstlisting}
      \item Iniciación por copia
\begin{lstlisting}
dispositivo d3{d1}; // Copia miembro a miembro
\end{lstlisting}
      \item Iniciación por defecto
\begin{lstlisting}
dispositivo d4{};
dispositivo d5;
\end{lstlisting}
    \end{itemize}
\end{itemize}
\end{frame}

\begin{frame}[fragile]
\begin{itemize}
  \item Si se usa iniciación por defecto:
    \begin{itemize}
      \item Variables globales $\rightarrow$ iniciación por defecto de miembros
\begin{lstlisting}
dispositivo d1; // id_serie="", capacidad=0
dispositivo d2{}; // id_serie="", capaciad=0
\end{lstlisting}
      \item Variables locales $\rightarrow$ iniciación por defecto de miembros de tipo clase
        \begin{itemize}
          \item Miembros de tipos primitivos sin iniciar.
        \end{itemize}
\begin{lstlisting}
void f() {
  dispositivo d3; // id_serie="", capacidad=?
  dispositivo d4{}; // id_sere="", capacidad=0
}
\end{lstlisting}
    \end{itemize}
  \item Se puede indicar valor por defecto para miembros.
\begin{lstlisting}
struct dispositivo {
  string id_serie{"sda1"};
  long long capacidad{1024};
};
\end{lstlisting}
\end{itemize}
\end{frame}

\subsection{Tipos con constructor}

\begin{frame}[fragile]{Invocación al constructor}
\begin{itemize}
  \item Sin un tipo tiene uno o más constructores, todas las definiciones de objetos 
        deben invocar algún constructor.
    \begin{itemize}
      \item El compilador deja de generar un constructor por defecto.
    \end{itemize}
\begin{lstlisting}
struct complejo {
  double real, imag;
  complejo(double r);
  complejo(double r, double i);
};

complejo c1; // Error
complejo c2{}; // Error
complejo c3{2.0}; // OK
complejo c4{2.0, 3.5}; // OK
complejo * pc = new complejo{2.0,3.5}; // OK
complejo v[] { {1.0,1.5}, {2.0, 2.0} }; // OK
vector<complejo> w { {1.0, 1.5}, {2.0, 2.0} }; // OK
\end{lstlisting}
\end{itemize}
\end{frame}

\begin{frame}[fragile]
\begin{itemize}
  \item Existe una sintaxis para forzar la invocación de un constructor.
    \begin{itemize}
      \item No invoca iniciación miembro a miembro o basadas en lista.
    \end{itemize}
\begin{lstlisting}
struct punto {
  double x, y;
};
punto p1(1.0, 1.0); // Error: No hay constructor
punto p2{1.0, 1.0}; // OK
punto p3(1.0); // Error: No hay constructor
punto p4{1.0}; // x=1.0, y=0.0
complejo c1(1.0, 1.0); // OK
complejo c2{1.0, 1.0}; // OK
complejo c3(1.0); // OK
complejo c4{1.0}; // OK
\end{lstlisting}
  \item Útil para discriminar entre constructores de \cppid{std::vector} de tipos numéricos.
\begin{lstlisting}
vector<int> v{10}; // Vector con el valor 10. v.size() == 1
vector<int> v(10); // Vector con 10 valores 0. v.size() == 10
\end{lstlisting}
\end{itemize}
\end{frame}

\subsection{Constructor por defecto}

\begin{frame}[fragile]
\begin{itemize}
  \item El \alert{constructor vacío} o \alert{constructor por defecto} es un constructor que no toma ningún parámetro.
\begin{lstlisting}
vector();
\end{lstlisting}
  \item Define que debe hacerse:
    \begin{itemize}
      \item Cuando se define un objeto sin valor inicial.
\begin{lstlisting}
vector v;
vector v{};
\end{lstlisting}
      \item Cuando se asigna memoria sin valor inicial
\begin{lstlisting}
vector * pv = new vector;
\end{lstlisting}
      \item El constructor vacío se genera si no hay ningún constructor.
        \begin{itemize}
          \item Se puede inhibir la generación del constructor vacío.
\begin{lstlisting}
vector() = delete;
\end{lstlisting}
          \item Se puede forzar la generación del constructor vacío.
\begin{lstlisting}
vector() = default;
\end{lstlisting}
        \end{itemize}
    \end{itemize}
\end{itemize}
\end{frame}

\begin{frame}[fragile]{Constructor por defecto y arrays}
\begin{itemize}
  \item El tipo base de un array debe tener constructor por defecto, si hace falta iniciar por defecto.
\begin{lstlisting}
struct punto {
  double x, y;
  punto();
};
struct complejo {
  double real, imag;
  complejo(double d, double i);
};

punto v[16]; // OK
complejo w[16]; // Error: No pueden iniciarse elementos
complejo z[] = { {1.0, 1.0}, {2.0, 2.0} }; // OK 
\end{lstlisting}
\end{itemize}
\end{frame}

\begin{frame}[fragile]{Constructor por defecto y vectores}
\begin{itemize}
  \item El tipo base de un vector debe tener constructor por defecto, si hace falta iniciar por defecto.
\begin{lstlisting}
struct punto {
  double x, y;
  punto();
};
struct complejo {
  double real, imag;
  complejo(double d, double i);
};

vector<punto> vp1; // OK. No necesita iniciar
vector<complejo> vc1; // OK. No necesita iniciar

vector<punto> vp2(16); // OK. 16 elementos con valor por defecto
vector<complejo> vc2(16); // Error: no puede iniciar 16 elementos
vector<complejo> vc3{ {1.0, 1.0}, {2.0, 2.0} }; // OK
vector<complejo> vc4{5, {1.0,1.0}}; // OK: 5 elementos iniciados
\end{lstlisting}
\end{itemize}
\end{frame}

\subsection{Conversión y constructor explícito}

\begin{frame}[fragile]{Constructor explícito}
\begin{itemize}
  \item Un constructor que toma un único argumento define una operación de conversión.
\begin{lstlisting}
class complejo {
  complejo(double r, double i);
  complejo(dobule r);
  // ...
};

complejo c = 3.5; // Conversión a complejo
\end{lstlisting}
  \item Sin embargo:
\begin{lstlisting}
vector v = 5; // Conversión de entero a vector
\end{lstlisting}
  \item Se puede forzar a que un constructor no actúe como operación de conversión:
\begin{lstlisting}
explicit vector(int n);
\end{lstlisting}
\end{itemize}
\end{frame}

\subsection{Constructor basado en lista}

\begin{frame}[fragile]{Lista de iniciación}
\begin{itemize}
  \item Un \alert{constructor basado en lista} es un constructor que toma un único argumento de tipo
        \cppid{std::initializer\_list}.
    \begin{itemize}
      \item Permiten definir la construcción de objetos a partir de una lista homogénea de valores.
    \end{itemize}
\begin{lstlisting}
lista l { 1, 2, 3, 4 }; // Invoca a lista(initializer_list<int>)
\end{lstlisting}
  \item Cualquier función puede tomar como parámetro una lista de iniciación.
    \begin{itemize}
      \item El argumento debe pasarse por valor.
    \end{itemize}
\begin{lstlisting}
int maximo(initializer_list<int> l);

void f() {
  x = maximo({1, 3, 4, 2});
  // ...
}
\end{lstlisting}
\end{itemize}
\end{frame}

\begin{frame}[fragile]{\texttt{initializer\_list}}
\begin{itemize}
  \item La clase \cppid{std::initializer\_list} ofrece funciones miembro para acceder a los valores de la lista:
    \begin{itemize}
      \item \cppid{begin()}: Puntero al principio de la lista.
      \item \cppid{end()}: Puntero al final de la lista (siguiente del último).
      \item \cppid{size()}: Tamaño de la lista.
    \end{itemize}
  \item Son suficientes para recorrer la lista:
\begin{lstlisting}
void imprime(std::initializer_list<int> l) {
  for (auto p=l.begin(); p!=l.end(); ++p) {
    cout << * p << endl;
  }
}
\end{lstlisting}
  \item O mejor:
\begin{lstlisting}
void imprime(std::initializer_list<int> l) {
  for (auto x : l) {
    cout << x << endl;
  }
}
\end{lstlisting}
\end{itemize}
\end{frame}

\subsection{Constructores para \texttt{vector}}

\begin{frame}{Constructores}
\begin{itemize}
  \item Constructor vacío
  \item Constructor con tamaño pasa a ser explícito.
  \item Constructor con lista de iniciación.
\end{itemize}
\begin{block}{vector.h}
\mode<presentation>{
\lstinputlisting[firstline=6,lastline=10]{03-copymove/vec2/vector.h}
}
\mode<article>{
\lstinputlisting[firstline=6,lastline=10]{03-copymove/vec2/vector.h}
}
\end{block}
\end{frame}

\begin{frame}
\begin{block}{vector.cpp}
\mode<presentation>{
\lstinputlisting[lastline=16]{03-copymove/vec2/vector.cpp}
}
\mode<article>{
\lstinputlisting{03-copymove/vec2/vector.cpp}
}
\end{block}
\end{frame}
