\section{Destructores}

\begin{frame}[fragile]{El problema de la desasignación}
\begin{itemize}
  \item Un tipo que adquiere recursos (ej. memoria) y no lo libera provoca un goteo de memoria.
  \item Solución simple:
    \begin{itemize}
      \item Función miembro de liberación.
    \end{itemize}
\begin{lstlisting}
class vector {
  // ...
  void libera() { delete []v; }
  // ...
};

int main() {
  vector v{5};
  // ...
  v.libera();
}
\end{lstlisting}
\end{itemize}
\end{frame}

\begin{frame}[fragile]{Función miembro de destrucción}
\begin{itemize}
  \item Un \alert{destructor} es una función miembro especial que se ejecuta 
        \textbf{de forma automática} cuando un objeto sale de alcance.
    \begin{itemize}
      \item No tiene tipo de retorno.
      \item No toma parámetros.
      \item Su nombre es el nombre de clase precedido del símbolo \cppkey{\~}.
    \end{itemize}
\begin{lstlisting}
class vector {
  // ...
  ~vector() { delete []v; }
  // ...
};
\end{lstlisting}
  \item Invocación automática.
\begin{lstlisting}
void f() {
  vector v{10};
  rellena(v);
  cout << v << endl;
} // Invoca al destructor de v
\end{lstlisting}
\end{itemize}
\end{frame}

\begin{frame}[fragile]{Destrucción generado por defecto}
\begin{itemize}
  \item En realidad todos los tipos tienen una función miembro destructor.
    \begin{itemize}
      \item El compilador genera una que es válida en la mayoría de los casos.
    \end{itemize}
  \item Dado un tipo sin destructor, el compilador \emph{sintetiza} uno:
    \begin{itemize}
      \item Invoca (recursivamente) al destructor de cada miembro.
      \item El destructor de un tipo primitivo es la operación nula.
    \end{itemize}
\end{itemize}
\begin{lstlisting}
struct estudiante {
  string nombre;
  vector<double> notas;
  // ...
};

void f() {
  estudiante e { "Carlos", {9.0, 9.5, 8.5} };
  cout << e.nombre << " -> " << e.notas[0] << endl;
} // Destrucción de e
\end{lstlisting}
\end{frame}
