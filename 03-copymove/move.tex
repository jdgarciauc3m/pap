\section{Operaciones de movimiento}

\subsection{Introducción}

\begin{frame}[fragile]{Intercambio de dos valores}
\begin{itemize}
  \item Función de intercambio para enteros
\begin{lstlisting}
void intercambia(int & x, int & y) {
  int aux{x};
  x = y;
  y = aux;
}

int a = 2, b = 4;
intercambia(a,b); // a=4, b=2
\end{lstlisting}
  \item Se realizan 3 copias de enteros.
\end{itemize}
\end{frame}

\begin{frame}[fragile]
\begin{itemize}
  \item Intercambio de vectores
\begin{lstlisting}
void intercambia(vector & v, vector & w) {
  vector aux{v};
  v = w;
  w = aux;
}
\end{lstlisting}
  \item Operaciones:
    \begin{enumerate}
      \item Construcción por copia de \cppid{aux} $\rightarrow$ 1 asignación de memoria.
      \item Asignación a \cppid{x} $\rightarrow$ 1 desasignación + 1 asignación de memoria.
      \item Asignación a \cppid{y} $\rightarrow$ 1 desasignación + 1 asignación de memoria.
      \item Destrucción de \cppid{tmp} $\rightarrow$ 1 desasignación de memoria
      \item 3 copias del vector elemento a elemento.
    \end{enumerate}
  \item Sin embargo:
    \begin{itemize}
      \item Sería suficiente con intercambiar tamaños y punteros
    \end{itemize}
\end{itemize}
\end{frame}

\begin{frame}[fragile]
\begin{itemize}
  \item Intercambio de vectores (si se tuviese acceso a implementación):
\begin{lstlisting}
void intercambia(vector & v, vector & w) {
  unsigned long aux_tam{v.tam};
  v.tam = w.tam;
  w.tam = aux.tam;
  double * aux_vec = v.vec;
  v.vec = w.vec;
  w.vec = aux.vec;
}
\end{lstlisting}
  \item Operaciones:
    \begin{enumerate}
      \item 0 asignaciones de memoria.
      \item 0 desasignaciones de memoria.
      \item 3 copias de enteros (3 palabras).
      \item 3 copias de puntero (3 palabras).
      \item 0 copias de contenidos de vector.
    \end{enumerate}
  \item Sin embargo:
    \begin{itemize}
      \item Sería necesario acceder a la implementación.
      \item Hay más casos de uso.
    \end{itemize}
\end{itemize}
\end{frame}

\begin{frame}[fragile]{Operaciones de movimiento}
\begin{itemize}
  \item Una operación de movimiento adquiere los recursos de un objeto y lo deja en un estado válido, pero vacío.
    \begin{itemize}
      \item Lo único que se espera de un objeto después es que pueda destruirse.
    \end{itemize}
  \item Operaciones de movimiento:
    \begin{itemize}
      \item Constructor de movimiento: Mueve el objeto origen al objeto creado.
      \item Asignación de movimiento: Mueve el objeto origen al objeto destino.
    \end{itemize}
  \item Orígenes de movimiento:
    \begin{itemize}
      \item Retorno de valores locales por copia.
      \item Variables a las que se aplica \cppid{std::move()}.
    \end{itemize}
\end{itemize}
\end{frame}

\begin{frame}[fragile]{Retorno de objetos locales}
\begin{itemize}
  \item Cuando una función devuelve una copia de un objeto local.
\begin{lstlisting}
string concatena(const string & a, const string & b) {
  string tmp = a + " -- " + b;
  return tmp; // Invoca una operación de movimiento.
}

x = concatena("Daniel", "Garcia");
\end{lstlisting}
  \item Se aprovecha el hecho de que \cppid{tmp} se va a destruir.
    \begin{itemize}
      \item Se transfieren los recursos al destino.
        \begin{itemize}
          \item Ejemplo: Copiando el valor del puntero.
        \end{itemize}
      \item Se deja al origen en un estado \emph{destruible}.
        \begin{itemize}
          \item Ejemplo: Haciendo el puntero \cppkey{nullptr}.
        \end{itemize}
    \end{itemize}
\end{itemize}
\end{frame}

\begin{frame}[fragile]{Aplicación explicita de movimiento}
\begin{itemize}
  \item Cuando se invoca sobre una variable la operación \cppid{std::move()}.
\begin{lstlisting}
void f() {
  string c = "Daniel";
  string d = "Garcia";
  d = std::move(c); // d == "Daniel", c == "Garcia" o vacía
} // Destructores
\end{lstlisting}
  \item Se informa que una variable puede moverse.
    \begin{itemize}
      \item Son las operaciones de movimiento (constructor) y (asignación) las
            que realizan el movimiento.
    \end{itemize}
\end{itemize}
\end{frame}

\subsection{Operaciones de movimiento}

\begin{frame}[fragile]{Constructor de movimiento}
\begin{itemize}
  \item El \alert{constructor de movimiento} se invoca cuando se construye un objeto
        a partir de una operación de movimiento.
    \begin{itemize}
      \item Toma un argumento de tipo \emph{referencia a r-valor}.
    \end{itemize}
\begin{lstlisting}
vector(vector &&);
\end{lstlisting}
  \item Se puede suprimir la generación automática del constructor de movimiento.
\begin{lstlisting}
vector(vector &&) = delete;
\end{lstlisting}
  \item Se puede forzar la generación automática del constructor de movimiento.
\begin{lstlisting}
vector(vector &&) = default;
\end{lstlisting}
  \item Se invoca en contextos como:
\begin{lstlisting}
vector v = f(); // vector f();
vector w = std::move(v);
\end{lstlisting}
\end{itemize}
\end{frame}

\begin{frame}[fragile]{Operador de asignación de movimiento}
\begin{itemize}
  \item El \alert{operador de asignación de movimiento} se invoca cuando se asigna un objeto
        a partir de una operación de movimiento.
    \begin{itemize}
      \item Toma un argumento de tipo \emph{referencia a r-valor}.
    \end{itemize}
\begin{lstlisting}
vector & operator=(vector &&);
\end{lstlisting}
  \item Se puede suprimir la generación automática del operador de asignación de movimiento.
\begin{lstlisting}
vector & operator=(vector &&) = delete;
\end{lstlisting}
  \item Se puede forzar la generación automática del operador de asignación de movimiento.
\begin{lstlisting}
vector & operator=(vector &&) = default;
\end{lstlisting}
  \item Se invoca en contextos como:
\begin{lstlisting}
v = f(); // vector f();
w = std::move(v);
\end{lstlisting}
\end{itemize}
\end{frame}

\begin{frame}{Implementación de constructor de movimiento}
\begin{block}{vector.cpp}
\lstinputlisting[firstline=35,lastline=41]{03-copymove/vec5/vector.cpp}
\end{block}
\end{frame}

\begin{frame}{Implementación de asignación de movimiento}
\begin{block}{vector.cpp}
\lstinputlisting[firstline=43,lastline=47]{03-copymove/vec5/vector.cpp}
\end{block}
\end{frame}
