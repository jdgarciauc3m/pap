\section{Evaluación}

\begin{frame}[t]{Sistema de evaluación}
\begin{itemize}
  \item Resumen:
  \vspace{1em}
    \begin{itemize}
      \item Examen final: 30\% de la calificación final.
        \begin{itemize}
          \item Incluye todos los contenidos (teoría, prácticas, y proyectos).
        \end{itemize}

      \item Evaluación continua: 70\% de la calificación final.
        \begin{itemize}
          \item Tests de evaluación: 30\% de la calificación final.
          \item Prácticas en grupo: 40\% de la calificación final.
        \end{itemize}
    \end{itemize}
  \vspace{1em}
  \begin{itemize}
    \item Convocatorias:
      \begin{itemize}
        \item Convocatoria ordinaria: Enero.
        \item Convocatoria extraordinaria: Junio.
      \end{itemize}
  \end{itemize}
\end{itemize}
\end{frame}

\begin{frame}[t]{Evaluación continua}
\begin{itemize}
  \item Obtener buen resultado en la evaluación continua es clave para superar la asignatura.
  \item Elementos:
    \begin{itemize}
      \item Tests de evaluación: 30\% de la calificación final.
      \item Prácticas: 40\% de la calificación final.
    \end{itemize}
  \vspace{1em}
  \item No has seguido la evaluación continua si:
    \begin{itemize}
      \item Obtienes menos de 3.5 en la media de todas las prácticas.
    \end{itemize}
\end{itemize}
\end{frame}

\begin{frame}[t]{Convocatoria ordinaria: Evaluación continua}
\begin{itemize}
  \item Si sigues el proceso de evaluación continua:
    \begin{itemize}
    \item Examen final: 30\%.
      \begin{itemize}
        \item Mínimo necesario: 3.0.
      \end{itemize}
    \item Tests de evaluación: 30\%
      \begin{itemize}
        \item Mínimo necesario: \alert{No hay mínimo}.
      \end{itemize}
    \item Prácticas: 40\%.
      \begin{itemize}
        \item Mínimo en la media de todas las prácticas: 3.5.
      \end{itemize}
    \item Si no logras algún mínimo, la media no se calcula y serás calificado como suspenso.
  \end{itemize}
  \item \alert{Bonus}:
    \begin{itemize}
      \item Se añadirá 1.5 puntos adicionales a la calificación final si:
        \begin{itemize}
          \item Obtienes al menos 7.0 puntos en la evaluación continua, y además
          \item obtienes al menos 6.0 puntos en el examen final.
        \end{itemize}
    \end{itemize}
\end{itemize}
\end{frame}

\begin{frame}[t]{Convocatoria ordinaria: Evaluación NO-continua}
\begin{itemize}
  \item Si no has seguido el proceso de evaluación continua:
    \begin{itemize}
      \item El examen final tiene un valor del 60\% de la calificación final.
      \item Necesitarás 8.33 en ele examen final para superar la asignatura.
    \end{itemize}
  \item \alert{CONSEJO}:
    \begin{itemize}
      \item Pon esfuerzo en seguir el proceso de evaluación continua.
    \end{itemize}
\end{itemize}
\end{frame}


\begin{frame}[t]{Convocatoria extraordinaria}
\begin{itemize}
  \item Examen extraordinario en el mes de junio.
  \vspace{1em}
  \item Normas:
    \begin{enumerate}
      \item Estudiantes que han completado el proceso de evaluación continua:
        \begin{itemize}
          \item El examen final vale el 30\% y la evaluación continua el otro 70\%.
          \item Solamente se aplica si la calificación en el examen es de al menos 3.5.
        \end{itemize}
      \item Estudiantes que no han completado el proceso de evaluación continua:
        \begin{itemize}
          \item El examen vale el 100\%.
        \end{itemize}
    \end{enumerate}
    \begin{itemize}
      \item A los estudiantes que hayan completado el proceso de evaluación continua se tomará la opción más favorable.
    \end{itemize}
\end{itemize}
\end{frame}

\begin{frame}[t]{Pruebas de evaluación}
\begin{itemize}
  \item \alert{MUY IMPORTANTE}:
    \begin{itemize}
      \item La no asistencia al examen final implica la calificación como NO-PRESENTADO, independientemente de cualquier otra calificación.
    \end{itemize}
\end{itemize}
\end{frame}
