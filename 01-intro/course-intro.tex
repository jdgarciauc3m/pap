\section{Visión general}

\subsection{Objetivos y competencias}

\begin{frame}[t]{Asignatura}
\begin{itemize}
  \item \textmark{Objetivo}: 
  que el estudiante sea capa de usar \textmark{lenguajes y técnicas de programación} que
  permitan obtener \textgood{altas prestaciones} en el contexto del desarrollo de 
  \textgood{software financiero}.

\end{itemize}
\end{frame}

\begin{frame}[t]{Resultados de aprendizaje}
\begin{itemize}
  \item Para alcanzar este objetivo, el estudiante profundizará en aspectos:
    \begin{itemize}
      \vspace{1.5em}
      \pause
      \item Conocer los principales lenguajes de programación que se utilizan para el desarrollo de software financiero.

      \vspace{1.5em}
      \pause
      \item Capacidad para implementar software para el sector financiero.

      \vspace{1.5em}
      \pause
      \item Conocimientos sobre la programación de altas prestaciones.
    \end{itemize}
\end{itemize}
\end{frame}


\subsection{Audiencia}

\begin{frame}{Audicencia}
\begin{itemize}
  \item Programación de Altas Prestaciones.
    \begin{itemize}
      \item Titulación: Master en Tecnologías de la Computación aplicadas al sector financiero.
      \item Tipo: Obligatoria.
      \item Curso: 1.
      \item Cuatrimestre: 1.
      \item Créditos: 6 ECTS.
    \end{itemize}

  \vfill
  \item Conocimientos previos:
    \begin{itemize}
      \item Programación.
      \item Arquitectura de Computadores.
      \item Estructuras de Datos.
      \item Sistemas Operativos.
    \end{itemize}
\end{itemize}
\end{frame}

\subsection{Programa de contenidos}

\begin{frame}[t]{Programa}
\begin{enumerate}
  \item Fundamentos de la Computación de Altas Prestaciones.
  \item Lenguajes de Programación.
  \item Gestión de Memoria.
  \item Programación genérica.
  \item Bibliotecas e interoperabilidad.
  \item Optimización de código.
  \item Análisis del rendimiento.
  \item Hilos y frameworks de concurrencia y paralelismo.
\end{enumerate}
\end{frame}
