\section{Tarea a realizar}

\subsection{Presentación general}

En esta práctica se desarrollará un tipo que permita representar una matriz
algebraica de números en coma flotante con las siguientes características:

\begin{itemize}

\item Construcción por defecto. Se construirá una matriz de 0 filas y 0
columnas.

\item Construcción a partir de un número de filas y un número de columnas. Este
constructor deberá asignar suficiente espacio de memoria para almacenar todos
los valores de la matriz.

\item Constructores de copia y de movimiento.

\item Operadores de asignación de copia y de movimiento.

\item Destructor. Debe liberar la memoria de la matriz.

\item Operador de acceso. Se sobrecargará el operador \cppid{()} tomando dos
parámetros para los índices \cppid{i} y \cppid{j} a los que se quiere acceder.
Los índices de la matriz empiezan en \cppid{0}.

\item Función miembro \cppid{filas()} permite obtener el número de filas de la
matriz.

\item Función miembro \cppid{columnas()} permite obtener el número de columnas
de la matriz.

\item Operador \cppid{<{}<} para realizar la escritura de una matriz.

\end{itemize}

En cuanto a la representación interna para la matriz se usarán los siguientes
datos miembro:

\begin{itemize}
\item \cppid{filas\_}: Número de filas de la matriz.
\item \cppid{columnas\_}: Número de columnas de la matriz.
\item \cppid{vec\_}: Un puntero a un bloque de memoria dinámica que contendrá
todos los valores de la matriz. Cada valor será un número representado en doble
precisión.
\end{itemize}

Por tanto el archivo de cabecera para la clase matriz tendrá la siguiente
estructura:

\begin{lstlisting}
#ifndef MATRIZ_H
#define MATRIZ_H

namespace alglin {

class matriz {
public:
  // Operaciones

private:
  int filas_;
  int columnas_;
  double * vec_;
};

}

#endif
\end{lstlisting}

\subsection{Constructor por defecto}

El constructor por defecto deberá crear una matriz de \cppid{0} fila y \cppid{0}
columnas.

Por tanto el siguiente código:

\begin{lstlisting}
void f() {
  using namespace std;
  matriz m;
  cout << "Filas. " << m.filas() << "\n";
  cout << "Columnas: " << m.columnas() << "\n";
}
\end{lstlisting}

Imprimirá
\begin{lstlisting}[style=terminal]
Filas: 0
Columnas: 0
\end{lstlisting}

\subsection{Constructor con tamaño}

Este constructor toma un número de filas y un número de columnas, y reserva
además espacio de memoria para los valores almancenados:

Por tanto el siguiente código:

\begin{lstlisting}
void f() {
  using namespace std;
  matriz m{3,4};
  cout << "Filas. " << m.filas() << "\n";
  cout << "Columnas: " << m.columnas() << "\n";
}
\end{lstlisting}

Imprimirá
\begin{lstlisting}[style=terminal]
Filas: 3
Columnas: 4
\end{lstlisting}

\subsection{Acceso a los valores de una matriz}

Para acceder a los valores de una matriz se sobrecargará el operador de llamada
a función (\cppkey{operator}\cppid{()}) tomando dos parámetros: las posiciones
\cppid{i} y \cppid{j} dentro de la matriz.

Se deberán definir dos versiones (una constante y otra no constante).

\begin{lstlisting}
double operator()(int i, int j) const;
double & operator() (int i, int j);
\end{lstlisting}

Como los datos se almacenan en un array unidimensional, la posición
bidimensional $(i,j)$ se transformará en la posición unidimensional $i \dot cols
+ j$, donde $cols$ es el número de columnas de la matriz.

Por tanto, el siguiente código:

\begin{lstlisting}
void f() {
  matriz m{3,4};
  m(1,2) = 1.5;
  std::cout << "m[1,2] = " << m(1,2) << "\n";
\end{lstlisting}

Imprimirá:

\begin{lstlisting}[style=terminal]
m[1,2] = 1.5
\end{lstlisting}

\subsection{Destrucción}

Debe comprobar que en todos los casos se libera adecuadamente la memoria,
para ello puede utilizar la herramienta \emph{valgrind} tal y como se ilustra en
el material de clase.

\subsection{Operaciones de copia}

Debe definirse un constructor de copia y un operador de asignación de copia, de
modo que sea posible realizar las siguientes acciones.

\begin{lstlisting}
void f() {
  matriz a{3,4};
  a(1,2) = 0.5;
  matriz b{a};
  std::cout << "b(1,2) =" << b(1,2) << "\n";

  matriz c{2,100};
  c = a;
  std::cout << "c.filas = " << c.filas() << "\n";
  std::cout << "c.columnas = " << c.columnas() << "\n";
  std::cout << "c(1,2) = " << c(1,2) << "\n";
}
\end{lstlisting}

Este código imprimirá:

\begin{lstlisting}[style=terminal]
b(1,2) = 0.5
c.filas = 3
c.columnas = 4
c(1,2) = 0.5
\end{lstlisting}

\subsection{Operaciones de movimiento}

Debe definirse un constructor de movimiento y un operador de asignación de
movimiento, de modo que sea posible realizar las siguientes acciones:

\begin{lstlisting}
void f() {
  matriz a{3,4};
  a(1,2) = 0.5;
  matriz b{std::move(a)};
  std::cout << "a.filas = " << a.filas() << "\n";
  std::cout << "a.columnas = " << a.columnas() << "\n";
  std::cout << "b.filas = " << b.filas() << "\n";
  std::cout << "b.columnas = " << b.columnas() << "\n";
  std::cout << "b(1,2) = " << b(1,2) << "\n";

  matriz c;
  c = std::move(b);
  std::cout << "b.filas = " << b.filas() << "\n";
  std::cout << "b.columnas = " << b.columnas() << "\n";
  std::cout << "c.filas = " << c.filas() << "\n";
  std::cout << "c.columnas = " << c.columnas() << "\n";
  std::cout << "c(1,2) = " << c(1,2) << "\n";
}
\end{lstlisting}

En este caso se imprimirá:

\begin{lstlisting}[style=terminal]
a.filas = 0
a.columnas = 0
b.filas = 3
b.columnas = 4
b(1,2) = 0.5;
b.filas = 0
b.columnas = 0
c.filas = 3
c.columnas = 4
c(1,2) = 0.5;
\end{lstlisting}

\subsection{Operadr de inserción}

El operador de inserción en flujos, escribirá la matriz fila a fila, separando
los valores dentro de una línea por espacios y terminando cada línea con un
salto de línea.

Por tanto el siguiente código:

\begin{lstlisting}
void f() {
  matriz a{2,2};
  a(0,0) = 1.0;
  a(0,1) = 2.0;
  a(1,1) = 3.0;
  std::cout << a;
}
\end{lstlisting}

Imprimirá:

\begin{lstlisting}[style=terminal]
1.0 2.0
0.0 3.0
\end{lstlisting}
