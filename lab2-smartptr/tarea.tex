\section{Tarea a realizar}

\subsection{Presentación general}

En esta práctica se continuará con el desarrollo de una clase \cppid{matriz}
iniciada en la práctica anterior, pero haciendo uso de punteros elegantes.

La representación interna estará constituida por los siguientes datos miembro:

\begin{itemize}
\item \cppid{filas\_}: Número de filas de la matriz.
\item \cppid{columnas\_}: Número de columnas de la matriz.
\item \cppid{vec\_}: Un puntero a un bloque de memoria dinámica que contendrá
todos los valores de la matriz. Cada valor será un número representado en doble
precisión. \textbad{Deberá usarse un puntero elegante}.
\end{itemize}

Deberá revisar el resto de miembros de la práctica anterior poniendo especial 
atención en las implementación de constructores, operadores de asignación y
destructor.

