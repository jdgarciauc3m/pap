\section{Introducción}

\subsection{Tipos genéricos}

\begin{frame}[fragile]{Tipos genéricos definidos por el usuario}
\vspace{-0.75em}
\begin{itemize}
  \item Una \textmark{plantilla} permite usar un tipo de datos que está
        parametrizado por otro tipo de datos.
\begin{lstlisting}
vector<double> v1;
vector<string> v2;
vector<fecha> v3;
vector<vector<string>> v4
\end{lstlisting}
\mode<presentation>{\pause}
  \item Una plantilla requiere que todas las definiciones sean visible en
        tiempo de compilación.
    \begin{itemize}
      \item Efecto práctico: Todas las definiciones en cabecera
    \end{itemize}
\end{itemize}
\begin{lstlisting}
template <class T>
class vector {
  // ...
};

template <typename U>
class lista {
  // ...
};
\end{lstlisting}
\end{frame}

\subsection{Un vector genérico}

\mode<presentation>{

\begin{frame}
\vspace{-0.5em}
\begin{block}{vector.h}
\lstinputlisting[lastline=21]{06-generic/vec1/vector.h}
\end{block}
\end{frame}

\begin{frame}
\begin{block}{vector.h}
\lstinputlisting[firstline=23,lastline=37]{06-generic/vec1/vector.h}
\end{block}
\end{frame}

\begin{frame}
\begin{block}{vector.h}
\lstinputlisting[firstline=39,lastline=54]{06-generic/vec1/vector.h}
\end{block}
\end{frame}

\begin{frame}
\begin{block}{vector.h}
\lstinputlisting[firstline=56,lastline=73]{06-generic/vec1/vector.h}
\end{block}
\end{frame}

\begin{frame}
\begin{block}{vector.h}
\lstinputlisting[firstline=75,lastline=90]{06-generic/vec1/vector.h}
\end{block}
\end{frame}

\begin{frame}
\begin{block}{vector.h}
\lstinputlisting[firstline=92,lastline=104]{06-generic/vec1/vector.h}
\end{block}
\end{frame}

\begin{frame}
\begin{block}{vector.h}
\lstinputlisting[firstline=106]{06-generic/vec1/vector.h}
\end{block}
\end{frame}

}

\mode<article>{

\begin{frame}
\begin{block}{vector.h}
\lstinputlisting{06-generic/vec1/vector.h}
\end{block}
\end{frame}

}

\begin{frame}
\begin{block}{main.cpp}
\lstinputlisting{06-generic/vec1/main.cpp}
\end{block}
\end{frame}

