\section{Plantillas y programación genérica}

\subsection{Programación genérica}

\begin{frame}[t]{Programación genérica}
\begin{itemize}
  \item Las plantillas son la base de la \textgood{programación genérica}.
\mode<presentation>{\vfill}
  \item \textmark{Programación genérica}: Desarrollar código que funcione
        con diversidad de tipos pasados como argumentos.
    \begin{itemize}
      \item Los tipos argumentos deben cumplir requisitos sintácticos y
            semánticos.
    \end{itemize}
\mode<presentation>{\vfill\pause}
  \item Los requisitos \textbad{no} se especifican explícitamente.
    \begin{itemize}
      \item Se deducen implícitamente.
    \end{itemize}
\mode<presentation>{\vfill\pause}
  \item Ejemplos para \cppid{vector<T>}:
    \begin{itemize}
      \item \cppid{T} debe poderse copiar $\rightarrow$ Constructor de copia.
      \item \cppid{T} debe poderse construir por defecto $\rightarrow$ Constructor por defecto.
    \end{itemize}
\end{itemize}
\end{frame}

\begin{frame}{Polimorfismo}
\begin{itemize}
  \item Tipos de polimorfismo:
    \begin{itemize}
      \item \textgood{Polimorfismo paramétrico}: Programación genérica.
      \item \textgood{Polimorfismo ad hoc}: Programación orientada a objetos.
    \end{itemize}
\mode<presentation>{\vfill\pause}
  \item Razones para elegir el polimorfismo paramétrico:
    \begin{itemize}
      \item \textmark{Rendimiento}: Dan lugar a código con menos sobrecarga 
            en tiempo de ejecución.
        \begin{itemize}
          \item La mayoría de las decisiones pueden tomarse en tiempo de compilación.
          \item Computación numérica.
          \item Tiempo real duro.
        \end{itemize}
 
      \item \textmark{Flexibilidad}: Permite la integración de bibliotecas concebidas 
            de forma separada.
        \begin{itemize}
          \item Tipos genéricos complejos.
          \item Muchos ejemplos en biblioteca estándar.
        \end{itemize}
    \end{itemize}
\end{itemize}
\end{frame}

\subsection{Valores como parámetro de plantilla}

\begin{frame}[fragile]{Parámetros de plantilla no-tipo}
\begin{itemize}
  \item Se puede definir una plantilla en la que alguno de sus parámetros no sea
        un tipo.
\begin{lstlisting}
template <typename T, int N>
struct array {
  // ...
};
\end{lstlisting}
  \item Los valores usados para instanciar un tipo concreto deben ser conocidos en
        tiempo de compilación.
\begin{lstlisting}
array<double,1024> v1; // OK

constexpr int max = 512;
array<int,max> v2; // OK

int tam = 256;
array<long,tam> v3; // Error
\end{lstlisting}
\end{itemize}
\end{frame}

\begin{frame}
\begin{block}{array.h}
\lstinputlisting{06-generic/arr1/array.h}
\end{block}
\end{frame}

\begin{frame}
\begin{block}{main.cpp}
\lstinputlisting{06-generic/arr1/main.cpp}
\end{block}
\end{frame}

\subsection{Plantillas de función}

\begin{frame}[fragile]{Plantillas de función}
\begin{itemize}
  \item Se puede definir una plantilla para definir de forma genérica
        una familia de funciones.
\begin{lstlisting}
template <typename O, typename C>
void imprime(O & fs, const C & c, string sep) {
  for (int i=0; i<c.tamanyo(); ++i) {
    fs << c[i] << sep;
  }
}
\end{lstlisting}
  \item En este caso los parámetros de la plantilla se deducen en su invocación.
\begin{lstlisting}
void f() {
  vector<string> v1 { "Daniel", "Carlos", "Jose" };
  array<double,4> v2 { 1.0, 2.0, 3.0, 4.0 };
  imprime(cout, v1, "\n"); // O:ostream, C:vector<string>
  imprime(cout, v2, " , "); // O:ostream, C:array<double,4>
}
\end{lstlisting}
\end{itemize}
\end{frame}
