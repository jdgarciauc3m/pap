\section{Tipos de bibliotecas}

\begin{frame}[t]{Bibliotecas}
\begin{itemize}
  \item Cada unidad de traducción se compila generando un archivo objeto.
    \begin{itemize}
      \item Un archivo objeto contiene el código compilado correspondiente
            a un archivo \cppid{.cpp}.
      \item Para generar un programa ejecutable hay que suministrar al
            editor de enlaces (\emph{linker}) la lista de archivos objeto.
    \end{itemize}
  \vfill
  \item Una biblioteca permite agrupar una colección de archivos objeto en
        una única entidad.
\end{itemize}
\end{frame}

\begin{frame}[t]{Bibliotecas estáticas}
\begin{itemize}
  \item \textgood{Biblioteca estática}: El código se incluye directamente
        en el ejecutable con el que se enlaza.
    \begin{itemize}
      \item \textmark{Windows}: Archivo \cppid{.lib}.
      \item \textmark{Linux}: Archivo \cppid{.a}.
    \end{itemize}

  \vfill
  \item \textgood{Observaciones}:
    \begin{itemize}
      \item Solamente hay que distribuir el ejecutable.
      \item Se asegura que se usa la versión correcta del código.
      \item Una copia de la biblioteca en cada ejecutable.
      \item Actualizar la biblioteca implica regenerar el ejecutable.
      \item Ejecutables muy grandes.
    \end{itemize}
\end{itemize}
\end{frame}

\begin{frame}[t,fragile]{Creación de biblioteca estática}
\begin{itemize}
  \item Generar un archivo objeto por cada \cppid{.cpp}.
\begin{lstlisting}[style=terminal]
g++ -c hello.cpp -o hello.o
g++ -c bye.cpp -o bye.o
\end{lstlisting}
  \item Agregar todos los archivos objeto.
    \begin{itemize}
      \item El archivo debe tener la forma \cppid{libXXXX.a}.
    \end{itemize}
\begin{lstlisting}[style=terminal]
ar rcs libgreet.a hello.o bye.o
\end{lstlisting}
  \item Compilar enlazando con la biblioteca estática
\begin{lstlisting}[style=terminal]
g++ main.cpp -o teststat -lgreet -L.
\end{lstlisting}
\end{itemize}
\end{frame}

\begin{frame}[t]{Bibliotecas dinámicas}
\begin{itemize}
  \item \textgood{Biblioteca dinámica}: El código no forma parte del ejecutable,
        sino que se carga de forma separada implícita o explícitamente.
    \begin{itemize}
      \item \textmark{Windows}: Archivo \cppid{.dll}.
      \item \textmark{Linux}: Archivo \cppid{.so}.
    \end{itemize}

  \vfill
  \item \textgood{Observaciones}:
    \begin{itemize}
      \item Varios programas pueden compartir el código de la biblioteca dinámica.
      \item Se puede actualizar la versión de la biblioteca dinámica sin recompilar el ejecutable.
      \item Se deben cargar dinámicamente o usar una pequeña biblioteca de importación.
    \end{itemize}
\end{itemize}
\end{frame}


\begin{frame}[t,fragile]{Creación de biblioteca dinámica}
\begin{itemize}
  \item Generar un archivo objeto por cada \cppid{.cpp}.
    \begin{itemize}
      \item Opción \cppid{-fPIC}.
    \end{itemize}
\begin{lstlisting}[style=terminal]
g++ -c -fPIC hello.cpp -o hello.o
g++ -c -fPICbye.cpp -o bye.o
\end{lstlisting}
  \item Enlazar archivos en una biblioteca dinámica.
    \begin{itemize}
      \item El archivo debe tener la forma \cppid{libXXXX.so.n}.
    \end{itemize}
\begin{lstlisting}[style=terminal,basicstyle=\tiny\ttfamily]
g++ -shared -fPIC -Wl,-soname,libgreet.so.1 -o libgreet.so.1 hello.o bye.o
ln -s libgreet.so.1 libgreet.so
\end{lstlisting}
  \item Compilar enlazando con la biblioteca dinámica. 
\begin{lstlisting}[style=terminal]
g++ main.cpp -o testdyn -L. -lgreet
\end{lstlisting}
\end{itemize}
\end{frame}

\begin{frame}[t,fragile]{Tamaño}
\begin{lstlisting}[style=terminal,basicstyle=\tiny\ttfamily]
ls -l
\end{lstlisting}
\begin{lstlisting}[style=terminal,basicstyle=\tiny\ttfamily]
-rw-rw-r-- 1 user user   93 nov 16 16:55 bye.cpp
-rw-rw-r-- 1 user user   53 nov 16 16:54 bye.h
-rw-rw-r-- 1 user user 2784 nov 16 17:14 bye.o
-rw-rw-r-- 1 user user   99 nov 16 16:49 hello.cpp
-rw-rw-r-- 1 user user   59 nov 16 16:48 hello.h
-rw-rw-r-- 1 user user 2784 nov 16 17:13 hello.o
-rw-rw-r-- 1 user user 5658 nov 16 16:57 libgreet.a
lrwxrwxrwx 1 user user   13 nov 16 17:21 libgreet.so -> libgreet.so.1
-rwxrwxr-x 1 user user 8952 nov 16 17:14 libgreet.so.1
-rw-rw-r-- 1 user user   62 nov 16 16:50 main.cpp
-rwxrwxr-x 1 user user 8600 nov 16 17:23 testdyn
-rwxrwxr-x 1 user user 9336 nov 16 17:02 teststat
\end{lstlisting}
\end{frame}

\begin{frame}[t,fragile]{Referencias}
\begin{lstlisting}[style=terminal,basicstyle=\tiny\ttfamily]
LD_LIBRARY_PATH=. ldd teststat
\end{lstlisting}
\begin{lstlisting}[style=terminal,basicstyle=\tiny\ttfamily]
	linux-vdso.so.1 =>  (0x00007fff969e6000)
	libstdc++.so.6 => /usr/lib/x86_64-linux-gnu/libstdc++.so.6 (0x00007fca74c73000)
	libc.so.6 => /lib/x86_64-linux-gnu/libc.so.6 (0x00007fca748aa000)
	libm.so.6 => /lib/x86_64-linux-gnu/libm.so.6 (0x00007fca745a0000)
	/lib64/ld-linux-x86-64.so.2 (0x00005634299e2000)
	libgcc_s.so.1 => /lib/x86_64-linux-gnu/libgcc_s.so.1 (0x00007fca74389000)
\end{lstlisting}

\vfill
\begin{lstlisting}[style=terminal,basicstyle=\tiny\ttfamily]
LD_LIBRARY_PATH=. ldd testdyn
\end{lstlisting}
\begin{lstlisting}[style=terminal,basicstyle=\tiny\ttfamily]
	linux-vdso.so.1 =>  (0x00007ffc70f60000)
	libgreet.so.1 => ./libgreet.so.1 (0x00007fddc6709000)
	libc.so.6 => /lib/x86_64-linux-gnu/libc.so.6 (0x00007fddc6309000)
	libstdc++.so.6 => /usr/lib/x86_64-linux-gnu/libstdc++.so.6 (0x00007fddc5f80000)
	/lib64/ld-linux-x86-64.so.2 (0x000055c63074c000)
	libm.so.6 => /lib/x86_64-linux-gnu/libm.so.6 (0x00007fddc5c77000)
	libgcc_s.so.1 => /lib/x86_64-linux-gnu/libgcc_s.so.1 (0x00007fddc5a60000)
\end{lstlisting}
\end{frame}

\begin{frame}[t,fragile]{Un ejecutable totalmente estático}
\begin{lstlisting}[style=terminal,basicstyle=\tiny\ttfamily]
g++ -static main.cpp -o testfullstat -lgreet -L.
\end{lstlisting}
\vfill
\begin{lstlisting}[style=terminal,basicstyle=\tiny\ttfamily]
ls -pla test*
\end{lstlisting}
\begin{lstlisting}[style=terminal,basicstyle=\tiny\ttfamily]
-rwxrwxr-x 1 user user    8600 nov 16 17:23 testdyn
-rwxrwxr-x 1 user user 2195360 nov 16 17:37 testfullstat
-rwxrwxr-x 1 user user    9336 nov 16 17:37 teststat
\end{lstlisting}
\end{frame}
