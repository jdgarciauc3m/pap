\section{Organización de bibliotecas}

\begin{frame}[t]{Espacios de nombres}
\begin{itemize}
  \item Un espacio de nombres permite organizar los nombres de las entidades que se definen.
    \begin{itemize}
      \item Evita tener que usar reglas de nombrado para dar nombres únicos a clases, tipos, funciones, \ldots
      \item Ejemplo: Toda la biblioteca estándar definida en espacio de nombres \cppid{std}.
    \end{itemize}

  \vfill
  \item Consejos:
    \begin{itemize}
      \item No definir nada en espacio de nombres global.
      \item Usar uno o más espacios de nombre por biblioteca.
    \end{itemize}
\end{itemize}
\end{frame}

\begin{frame}[t,fragile]{Ejemplo}
\begin{columns}

\column{.5\textwidth}

\begin{block}{hora.h}
\begin{lstlisting}
#ifndef CALENDAR_HORA_H
#define CALENDAR_HORA_H

namespace calendar {

  class hora {
    // ...
  };

}

#endif
\end{lstlisting}
\end{block}

\column{.5\textwidth}

\begin{block}{fecha.h}
\begin{lstlisting}
#ifndef CALENDAR_FECHA_H
#define CALENDAR_FECHA_H

namespace calendar {

  class fecha {
    // ...
  };

}

#endif
\end{lstlisting}
\end{block}

\end{columns}
\end{frame}

\begin{frame}[t,fragile]{Espacios de nombre anonimos}
\begin{itemize}
  \item Un espacio de nombres anónimo define un espacio de nombres que solamente es visible
        por la unidad de traducción actual.
    \begin{itemize}
      \item Los símbolos definidos en él tienen enlace interno y no son accesibles desde otras
            unidades de traducción.
      \item Útil para definir funciones, clases, variables internas a una unidad de traducción.
    \end{itemize}
\end{itemize}
\begin{columns}

\column{.5\textwidth}

\begin{block}{modulo.h}
\begin{lstlisting}
#ifndef MODULO_H
#define MODULO_H

void f();

#endif
\end{lstlisting}
\end{block}

\column{.5\textwidth}

\begin{block}{modulo.cpp}
\begin{lstlisting}
#include "modulo.h"

namespace {
void g() { /*...*/ }
}
void f() { /*...*/ }
\end{lstlisting}
\end{block}

\end{columns}
\end{frame}
