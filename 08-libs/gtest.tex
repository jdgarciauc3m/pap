\section{Pruebas unitaras: Google Test}

\subsection{Introducción a Google Test}

\begin{frame}[t]{¿Qué es Google Test?}
\begin{itemize}
  \item Es una biblioteca para realizar pruebas unitarias en C++.
  
  \vfill
  \item \textenum{Características}:
    \begin{itemize}
      \item Un framework tipo \textmark{xUnit}.
      \item Descubrimiento de tests.
      \item Variedad de aserciones.
      \item Aserciones definidas por el usuario.
      \item \emph{Death tests}.
      \item Fallos fatales y no fatales.
      \item Tests parametrizados por valor.
      \item Tests parametrizados por tipo.
      \item Múltiples opciones de ejecución.
      \item Generación de informes en XML.
    \end{itemize}

\end{itemize}
\end{frame}

\begin{frame}[t]{Uso}
\begin{itemize}
  \item \textenum{Plataformas}:
    \begin{itemize}
      \item Linux
      \item Mac OS X
      \item Windows
      \item Cygwin
      \item MinGW
      \item Windows Mobile
      \item Symbian
    \end{itemize}
  \vfill
  \item Algunos proyectos que lo \textenum{usan}:
    \begin{itemize}
      \item Proyecto \textmark{Chromium}.
      \item Familia de compiladores \textmark{LLVM}.
      \item \textmark{Protocol Buffers}.
      \item \textmark{OpenCV}.
    \end{itemize}
\end{itemize}
\end{frame}

\begin{frame}[t]{Principios}
\begin{enumerate}
  \item Las pruebas deberían ser \textmark{independientes} y \textmark{repetibles}.
  \item Las pruebas deberían poder estar bien \textmark{estructuradas}.
  \item Las pruebas deberían ser \textmark{portables} y \textmark{reusables}.
  \item Ante un fallo en una prueba se debería poder obtener la \textmark{máxima información}.
  \item No debería ser necesario escribir \textmark{código innecesario} para pruebas.
  \item Las pruebas deberían ser \textmark{rápidas}.
\end{enumerate}
\end{frame}

\begin{frame}[t,fragile]{¿Como obtener GTest?}
\begin{itemize}
  \item Repositorio en GitHub.
    \begin{itemize}
      \item \url{https://github.com/google/googletest}
    \end{itemize}
  \vfill
  \item Se puede obtener como paquete (Ubuntu/Linux).
    \begin{itemize}
      \item Paquete: \cppid{libgtest-dev}.
      \item \textbad{Importante}: No instala binarios, solamente fuentes.
    \end{itemize}
\end{itemize}
\begin{lstlisting}[style=terminal]
$ git clone git@github.com:google/googletest.git
$ cd googletest
$ mkdir build
$ cd build
$ cmake ..
$ make -j3
$ sudo make install
\end{lstlisting}
\end{frame}

\subsection{Ejemplo 1: vectint}

\begin{frame}[t]{Un vector de enteros}
\begin{itemize}
  \item Comencemos con una clase sencilla
  \item Un vector de enteros con comportamiento mínimo:
    \begin{itemize}
      \item Un vector de enteros con tamaño (\cppid{size()}) y capacidad (\cppid{capacity()}).
      \item Versión mínima que soporte copia y movimiento.
      \item Acceso \cppkey{operator[]}.
      \item Modificación de capacidad (\cppid{reserve()}). 
      \item Modificación de tamaño (\cppid{resize()}). 
    \end{itemize}
\end{itemize}
\end{frame}

\begin{frame}[t]{Interfaz}
\begin{block}{include/vectint.h}
\lstinputlisting[lastline=15]{08-libs/examples/vector1/include/vectint.h}
\end{block}
\end{frame}

\begin{frame}[t]{Interfaz}
\begin{block}{include/vectint.h}
\lstinputlisting[firstline=16,lastline=26]{08-libs/examples/vector1/include/vectint.h}
\end{block}
\end{frame}

\begin{frame}[t]{Interfaz}
\begin{block}{include/vectint.h}
\lstinputlisting[firstline=27]{08-libs/examples/vector1/include/vectint.h}
\end{block}
\end{frame}

\begin{frame}[t]{Implementación}
\begin{block}{src/vectint.cpp}
\lstinputlisting[lastline=12]{08-libs/examples/vector1/src/vectint.cpp}
\end{block}
\end{frame}

\begin{frame}[t]{Implementación}
\begin{block}{src/vectint.cpp}
\lstinputlisting[firstline=13,lastline=27]{08-libs/examples/vector1/src/vectint.cpp}
\end{block}
\end{frame}

\begin{frame}[t]{Implementación}
\begin{block}{src/vectint.cpp}
\lstinputlisting[firstline=29,lastline=43]{08-libs/examples/vector1/src/vectint.cpp}
\end{block}
\end{frame}

\begin{frame}[t]{Implementación}
\begin{block}{src/vectint.cpp}
\lstinputlisting[firstline=44]{08-libs/examples/vector1/src/vectint.cpp}
\end{block}
\end{frame}

\begin{frame}[t]{Programa de prueba}
\begin{columns}[T]

\column{.5\textwidth}
\begin{block}{samples/main1.cpp}
\lstinputlisting[lastline=13]{08-libs/examples/vector1/samples/main1.cpp}
\end{block}

\column{.5\textwidth}
\begin{block}{samples/main1.cpp}
\lstinputlisting[firstline=14]{08-libs/examples/vector1/samples/main1.cpp}
\end{block}

\end{columns}
\end{frame}

\subsection{Usando CMake}

\begin{frame}[t]{¿Qué es CMake}
\begin{itemize}
  \item Es un conjunto de herramientas que permite construir, probar
        y empaquetar software.
  
    \begin{itemize}
      \item A partir de un proyecto CMake se pueden generar scripts de
            construcción para cada plataforma.
    \end{itemize}

  \item Generadores:
    \begin{itemize}
      \item Makefiles (Unix, Borland, MinGW, NMake).
      \item Ninja.
      \item Visual Studio.
      \item Xcode.
      \item Varios IDE:  CodeBlocks, CodeLite, EclipseCDT, KDevelop.
      \item IDE con soporte nativo: CLion, Visual Studio 2017 RC1.
    \end{itemize}
\end{itemize}
\end{frame}

\begin{frame}[t]{CMake raíz}
\vspace{-1em}
\begin{block}{CMakeLists.txt}
\lstinputlisting[basicstyle=\tiny]{08-libs/examples/vector1/CMakeLists.txt}
\end{block}
\end{frame}

\begin{frame}[t]{CMake para src}
\begin{block}{src/CMakeLists.txt}
\lstinputlisting{08-libs/examples/vector1/src/CMakeLists.txt}
\end{block}
\end{frame}

\begin{frame}[t]{CMake para samples}
\begin{block}{samples/CMakeLists.txt}
\lstinputlisting{08-libs/examples/vector1/samples/CMakeLists.txt}
\end{block}
\end{frame}

\begin{frame}[t,fragile]{Construyendo con CMake}
\begin{lstlisting}[style=terminal,basicstyle=\tiny\ttfamily]
jdgarcia@gavilan:~/vector1$ mkdir build
jdgarcia@gavilan:~/vector1$ cd build/
jdgarcia@gavilan:~/vector1/build$ cmake ..
-- The C compiler identification is GNU 6.2.0
-- The CXX compiler identification is GNU 6.2.0
-- Check for working C compiler: /usr/bin/cc
-- Check for working C compiler: /usr/bin/cc -- works
-- Detecting C compiler ABI info
-- Detecting C compiler ABI info - done
-- Check for working CXX compiler: /usr/bin/c++
-- Check for working CXX compiler: /usr/bin/c++ -- works
-- Detecting CXX compiler ABI info
-- Detecting CXX compiler ABI info - done
-- Configuring done
-- Generating done
-- Build files have been written to: /home/jdgarcia/vector1/build
\end{lstlisting}
\end{frame}

\begin{frame}[t,fragile]{¿Y si quiero elegir el compilador?}
\begin{lstlisting}[style=terminal,basicstyle=\tiny\ttfamily]
jdgarcia@gavilan:~//vector1/build$ cmake .. -DCMAKE_CXX_COMPILER=clang++ -DCMAKE_C_COMPILER=clang
-- Configuring done
You have changed variables that require your cache to be deleted.
Configure will be re-run and you may have to reset some variables.
The following variables have changed:
CMAKE_C_COMPILER= clang
CMAKE_CXX_COMPILER= clang++

-- The C compiler identification is Clang 3.8.0
-- The CXX compiler identification is Clang 3.8.0
-- Check for working C compiler: /usr/bin/clang
-- Check for working C compiler: /usr/bin/clang -- works
-- Detecting C compiler ABI info
-- Detecting C compiler ABI info - done
-- Check for working CXX compiler: /usr/bin/clang++
-- Check for working CXX compiler: /usr/bin/clang++ -- works
-- Detecting CXX compiler ABI info
-- Detecting CXX compiler ABI info - done
-- Configuring done
-- Generating done
-- Build files have been written to: /home/jdgarcia//vector1/build
\end{lstlisting}
\end{frame}

\begin{frame}[t,fragile]{Proceso de compilación}
\begin{lstlisting}[style=terminal,basicstyle=\tiny\ttfamily]
jdgarcia@gavilan:~/vector1/build$ make
Scanning dependencies of target dcl
[ 50%] Building CXX object src/CMakeFiles/dcl.dir/vectint.cpp.o
Linking CXX static library libdcl.a
[ 50%] Built target dcl
Scanning dependencies of target test1
[100%] Building CXX object samples/CMakeFiles/test1.dir/main1.cpp.o
Linking CXX executable ../test1
[100%] Built target test1
\end{lstlisting}
\end{frame}

\subsection{Mis primeras pruebas}

\begin{frame}[t]{Casos de pruebas y pruebas}
\begin{itemize}
  \item En GTest podemos definir pruebas y agruparlas en casos de prueba.
    \begin{itemize}
      \item Hacemos uso de la macro \cppid{TEST}.
      \item Caso de prueba (\emph{test case}): Prueba elemental.
      \item Prueba: Cada una de las pruebas de un caso de prueba.
    \end{itemize}

  \vfill
  \item Realización de comprobaciones:
    \begin{itemize}
      \item \cppid{EXPECT\_EQ}: Comprobación no fatal.
      \item \cppid{ASSERT\_EQ}: Comprobación no fatal.
    \end{itemize}
\end{itemize}
\end{frame}

\begin{frame}[t]{Probando la construcción}
\begin{block}{utest/vectint\_constructor.cpp}
\lstinputlisting{08-libs/examples/vector2/utest/vectint_constructor.cpp}
\end{block}
\end{frame}

\begin{frame}[t,fragile]{Ejecutando las pruebas}
\begin{lstlisting}[style=terminal]
jdgarcia@gavilan:~/vector2/build$ ./vectint_utest 
Running main() from gtest_main.cc
[==========] Running 2 tests from 1 test case.
[----------] Global test environment set-up.
[----------] 2 tests from vectint_constructor
[ RUN      ] vectint_constructor.empty
[       OK ] vectint_constructor.empty (0 ms)
[ RUN      ] vectint_constructor.sized
[       OK ] vectint_constructor.sized (0 ms)
[----------] 2 tests from vectint_constructor (0 ms total)

[----------] Global test environment tear-down
[==========] 2 tests from 1 test case ran. (0 ms total)
[  PASSED  ] 2 tests.
\end{lstlisting}
\end{frame}

\begin{frame}[t,fragile]{Comprobando la copia}
\begin{columns}[T]

\column{.5\textwidth}
\begin{block}{utest/vectint\_copy.cpp}
\lstinputlisting[lastline=15]{08-libs/examples/vector2/utest/vectint_copy.cpp}
\end{block}

\column{.5\textwidth}
\begin{block}{utest/vectint\_copy.cpp}
\lstinputlisting[firstline=17]{08-libs/examples/vector2/utest/vectint_copy.cpp}
\end{block}
\end{columns}
\end{frame}

\begin{frame}[t]{Configurando la compilación de las pruebas}
\begin{block}{utest/CMakeLists.txt}
\lstinputlisting[basicstyle=\tiny]{08-libs/examples/vector2/utest/CMakeLists.txt}
\end{block}
\end{frame}

\begin{frame}[t,fragile]{Integrando las pruebas en CMake}
\begin{itemize}
  \item Basta añadir una invocación a \cppid{enable\_testing()} antes de
        incluir el directorio de tests.
\end{itemize}

\vfill
\begin{block}{CMakeLists.txt}\small
\begin{lstlisting}
#...
enable_testing()
add_subdirectory(utest)
\end{lstlisting}
\end{block}

\vfill
\begin{itemize}
  \item Se genera nuevo objetivo para ejecutar las pruebas
    \begin{itemize}
      \item \cppkey{make test}.
      \item \cppkey{ctest}.
    \end{itemize}
\end{itemize}
\end{frame}

\begin{frame}[t,fragile]{Ejecutando las pruebas}
\begin{lstlisting}[style=terminal]
jdgarcia@gavilan:~/github-repos/examples/gtesttalk/vector2/build$ make test
Running tests...
Test project /home/jdgarcia/github-repos/examples/gtesttalk/vector2/build
    Start 1: vectint_copy.copy_construct
1/4 Test #1: vectint_copy.copy_construct ......   Passed    0.00 sec
    Start 2: vectint_copy.copy_assign
2/4 Test #2: vectint_copy.copy_assign .........   Passed    0.00 sec
    Start 3: vectint_constructor.empty
3/4 Test #3: vectint_constructor.empty ........   Passed    0.00 sec
    Start 4: vectint_constructor.sized
4/4 Test #4: vectint_constructor.sized ........   Passed    0.00 sec

100% tests passed, 0 tests failed out of 4

Total Test time (real) =   0.01 sec
\end{lstlisting}
\end{frame}



