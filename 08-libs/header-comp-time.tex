\section{Includes y tiempo de compilación}

\begin{frame}[t]{Modelo de inclusión}
\begin{itemize}
  \item El modelo de inclusión de archivos cabecera de C/C++:
    \begin{itemize} 
      \item Realiza una sustitución de cada directiva por el
            contenido del archivo incluido.
      \item Posteriormente se compila el archivo resultante.
    \end{itemize}

  \vfill
  \item Observaciones:
    \begin{itemize}
      \item Se compilan muchas líneas de código.
      \item Las inclusiones innecesarias retrasan la compilación.
    \end{itemize}
\end{itemize}
\end{frame}

\begin{frame}[t,fragile]{Inclusión múltiple}
\begin{itemize}
  \item Si se incluye un archivo de cabecera múltiples veces se generan errores
        al intentar definirse dos veces la misma clase, función, \ldots

  \item ¿Cómo evitar la inclusión múltiple?
    \begin{itemize}
      \item Usando guardas de include
    \end{itemize}
\end{itemize}

\begin{columns}

\column{.5\textwidth}
\begin{block}{elemento.h}
\begin{lstlisting}[basicstyle=\tiny\ttfamily]
#ifndef ELEMENTO_H
#define ELEMENTO_H

class elemento {
  // Detalles...
};

#endif
\end{lstlisting}
\end{block}

\pause
\column{.5\textwidth}
\begin{block}{documento.h}
\begin{lstlisting}[basicstyle=\tiny\ttfamily]
#include "elemento.h
#include <iostream>
// ...
#include "elemento.h" // Sin problemas
\end{lstlisting}
\end{block}

\end{columns}
\end{frame}

\begin{frame}[t,fragile]{Tiempo de compilación}
\lstinputlisting{08-libs/headcmptime/main.cpp}
\vfill
\begin{lstlisting}[style=terminal]
gcc -E main.cpp | wc -l
\end{lstlisting}
\begin{lstlisting}[style=terminal]
27236
\end{lstlisting}
\end{frame}

\begin{frame}[t,fragile]{Tiempo de compilación}
\lstinputlisting{08-libs/headcmptime/main2.cpp}
\vfill
\begin{lstlisting}[style=terminal]
gcc -E main.cpp | wc -l
\end{lstlisting}
\begin{lstlisting}[style=terminal]
48381
\end{lstlisting}
\end{frame}

\begin{frame}[t,fragile]{Declaraciones adelantadas}
\begin{itemize}
  \item Hay casos en que el compilador no necesita ver una clase.
    \begin{itemize}
      \item Es suficiente con una referencia adelantada.
    \end{itemize}
\end{itemize}
\begin{columns}

\column{.5\textwidth}
\begin{block}{elemento.h}
\begin{lstlisting}[basicstyle=\tiny\ttfamily]
class elemento {
  // Detalles...
};
\end{lstlisting}
\end{block}

\begin{block}{documento.h}
\begin{lstlisting}[basicstyle=\tiny\ttfamily]
#include "elemento.h

class docuemnto {
public:
  //...
  void store(elemento & e);
};
\end{lstlisting}
\end{block}

\pause
\column{.5\textwidth}
\begin{block}{elemento.h}
\begin{lstlisting}[basicstyle=\tiny\ttfamily]
class elemento {
  // Detalles...
};
\end{lstlisting}
\end{block}

\begin{block}{documento.h}
\begin{lstlisting}[basicstyle=\tiny\ttfamily]
class elemento;

class docuemnto {
public:
  //...
  void store(elemento & e);
};
\end{lstlisting}
\end{block}


\end{columns}
\end{frame}

\begin{frame}[t]{Declaraciones versus definiciones}
\begin{itemize}
  \item ¿Cuándo es suficiente una referencia adelantada?
    \begin{itemize}
      \item Cuando no es necesario conocer el tamaño de un objeto de ese tipo; y
      \item no se necesita saber qué operaciones ofrece un objeto de ese tipo.
    \end{itemize} 

  \vfill
  \item Ejemplos en que la declaración adelantada de un tipo \cppid{T} no es suficiente:
    \begin{itemize}
      \item Si se define una clase con un dato miembro de tipo \cppid{T}.
      \item Si se define una variable de tipo \cppid{T}.
      \item Si se declara una función que toma un valor de tipo \cppid{T} por valor.
      \item Si se declara una función que devuelve un valor de tipo \cppid{T} por valor.
    \end{itemize}
\end{itemize}
\end{frame}

\begin{frame}[t]{Optimización de includes}
\begin{itemize}
  \item Reducir al máximo los ficheros que se incluyen en cualquier archivo.
    \begin{itemize}
      \item Reduce el número de archivos que se abren.
    \end{itemize}
  \vfill
  \item Si se puede elegir entre incluir una cabecera en un \cppid{.h} o en un \cppid{.cpp}
        es mejor optar por hacerlo en el \cppid{.cpp}.
    \begin{itemize}
      \item El impacto se propaga más si se hace un un \cppid{.h}.
    \end{itemize}
  \vfill
  \item Si es necesario definir archivos de cabecera con conjuntos de declaraciones adelantadas
        relacionadas entre si.
    \begin{itemize}
      \item Ejemplo: \cppid{<iosfwd>} contiene declaraciones adelantadas para \cppid{<iostream>}.
    \end{itemize}
\end{itemize}
\end{frame}

\begin{frame}[t]{Ejemplo}
\begin{columns}

\column{.52\textwidth}
\begin{block}{fecha.h}
\lstinputlisting[basicstyle=\tiny\ttfamily]{08-libs/iosfwd/fecha.h}
\end{block}

\column{.48\textwidth}
\begin{block}{fecha.cpp}
\lstinputlisting[basicstyle=\tiny\ttfamily]{08-libs/iosfwd/fecha.cpp}
\end{block}

\end{columns}
\end{frame}

\begin{frame}{Bibliotecas header-only}
\begin{itemize}
  \item Una biblioteca \emph{header-only} es una biblioteca que está autocontenida en uno o más
        archivos de cabecera.
    \begin{itemize}
      \item Ejemplos: Algunos componentes de \emph{Boost}; \emph{includeOS}; \emph{Eigen}; \ldots
    \end{itemize}

  \vfill
  \item \textgood{Ventajas}:
    \begin{itemize}
      \item No necesitan compilarse.
      \item No necesitan empaquetarse o instalarse.
    \end{itemize}

  \vfill
  \item \textgood{Desventajas}:
    \begin{itemize}
      \item Tiempos de compilación más largos.
      \item Cualquier cambio requiere recompilar todas las unidades de traducción que la incluyen.
      \item Posible explosión de tamaño de código binario.
    \end{itemize}
\end{itemize}
\end{frame}

\begin{frame}[t]{Cabeceras precompiladas}
\begin{itemize}
  \item \textmark{Idea básica}: Tener precompilado el código derivado de un archivo de cabecera.
    \begin{itemize}
      \item Útil para ahorrar tiempo si la cabecera se incluye desde muchos sitios.
    \end{itemize}
\end{itemize}
\end{frame}

%\begin{frame}{El futuro: Módulos}
%\end{frame}

% TODO: Idom PIMPL
