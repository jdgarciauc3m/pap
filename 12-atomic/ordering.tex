\section{Relaciones de ordenación}


\begin{frame}[t]{Relación \textbf{sincroniza-con}}
\begin{itemize}
  \item \textgood{Relación} entre operaciones sobre \textmark{tipos atómicos}.
  \vfill
  \item Una \textmark{escritura} sobre un valor atómico 
        \textgood{sincroniza-con} 
        una \textmark{lectura} sobre ese valor atómico \textmark{que lee el valor}:
    \mode<presentation>{\vfill}
    \begin{enumerate}[i]
      \item Almacenado por \textmark{esa escritura}.
      \item Almacenado por una \textmark{escritura subsiguiente} del mismo hilo que realizó la escritura.
      \item Almacenado por una \textmark{secuencia} de operaciones \textmark{read-modify-write}
            sobre el valor de cualquier hilo en la que la primera operación 
            leyó el valor almacenado por la escritura.
    \end{enumerate}
\end{itemize}
\end{frame}


\begin{frame}[t]{Relación \textbf{ocurre-antes}}
\begin{itemize}
  \item Especifica qué operación \textgood{ve los efectos} de otra operación.

  \mode<presentation>{\vfill\pause}
  \item Dentro de un hilo, una operación \textgood{ocurre-antes} que otra si aparece en una 
        \textmark{sentencia anterior}.
    \begin{itemize}
      \item \textbad{No hay orden entre dos operaciones que aparecen en la misma sentencia}.
    \end{itemize}

  \mode<presentation>{\vfill\pause}
  \item Entre dos hilos, una operación en un hilo \textgood{ocurre-antes} que otra operación de otro hilo si:
    \begin{enumerate}[i]
      \item Existe una relación \textgood{sincroniza-con} entre ambas.
      \item Existe una cadena de relaciones \textgood{ocurre-antes} y \textgood{sincroniza-con} entre ellas.
    \end{enumerate}
\end{itemize}
\end{frame}

\begin{frame}[fragile]{Ordenación: Consistencia secuencial}
\begin{columns}[T]

\column{.5\textwidth}

\begin{block}{Ejemplo}
\begin{lstlisting}
std::vector<int> v; 
std::atomic_bool f(false); 

void escritor() { 
  v.push_back(1); // #1
  f = true; // #2
}

void lector() { 
  while(!f.load()) { // #3
    std::this_thread::sleep(
      std::milliseconds(1)); 
  }
  std::cout << v[0] << std::endl; // #4
} 
\end{lstlisting}
\end{block}

\column{.5\textwidth}

\input{12-atomic/seqconsist-ej1.tkz}

\mode<presentation>{\vfill\pause}
\begin{itemize}
  \item Único resultado posible: \cppid{v[0] == 1}.
\end{itemize}
\end{columns}

\end{frame}

