\section{Tipos atómicos}

\begin{frame}[t]{Operaciones atómicas}
\begin{itemize}
\item Son \textmark{operaciones indivisibles}.
    \begin{itemize}

      \mode<presentation>{\vfill\pause}
      \item Si un hilo realiza una \textmark{lectura atómica} de una variable y 
            otro una \textmark{escritura atómica} de la misma variable y no hay 
            \textmark{más hilos accediendo}:
        \begin{itemize}
          \item La lectura devuelve el \textmark{valor previo} a la escritura o el \textmark{valor escrito}.
        \end{itemize}

      \mode<presentation>{\vfill\pause}
      \item Si alguna de las operaciones (lectura o escritura) es \textmark{no atómica} 
            el \textbad{comportamiento no está definido}.
        \begin{itemize}
          \item Se puede obtener un valor que no sea ni el anterior ni el posterior.
        \end{itemize}
    \end{itemize}
\end{itemize}
\end{frame}

\begin{frame}[t]{Tipos atómicos}
\begin{itemize}
  \item El tipo genérico \cppid{atomic<T>} permite definir variables atómicas para el tipo \cppid{T}, donde \cppid{T} es:
    \begin{itemize}
      \item Un tipo integral.
      \item Un tipo puntero.
      \item El tipo \cppid{bool}.
      \item No está definido para tipos reales (\cppid{float}, \cppid{double}).
      \item También para tipos definidos por el usuario que cumplan con algunas restricciones.
    \end{itemize}
  \mode<presentation>{\vfill\pause}
  \item Todos los tipos atómicos tienen un miembro \cppid{is\_lock\_free()}.
    \begin{itemize}
      \item Determina si su implementación es libre de cerrojos.
    \end{itemize}
  \mode<presentation>{\vfill\pause}
  \item Además existe un tipo \cppid{atomic\_flag}:
    \begin{itemize}
      \item El único que garantiza ser libre de cerrojos.
    \end{itemize}
\end{itemize}
\end{frame}

\begin{frame}[t]{Operaciones sobre tipos atómicos}
\begin{itemize}
  \item Las operaciones sobre atómicos pueden especificar opcionalmente un ordenamiento de memoria.
    \begin{itemize}
      \item Por defecto \cppid{memory\_order\_seq\_cst}.
    \end{itemize}
  \item Operaciones de almacenamiento:
    \begin{itemize}
      \item \cppid{memory\_order\_relaxed}, \cppid{memory\_order\_release}, \cppid{memory\_order\_seq\_cst}.
    \end{itemize}
  \item Operaciones de lectura:
    \begin{itemize}
      \item \cppid{memory\_order\_relaxed}, \cppid{memory\_order\_consume}, \cppid{memory\_order\_acquire}, \cppid{memory\_order\_seq\_cst}
    \end{itemize}
  \item Operaciones lectura-modificación-escritura:
    \begin{itemize}
      \item \cppid{memory\_order\_relaxed}, \cppid{memory\_order\_consume}, \cppid{memory\_order\_acquire}, \cppid{memory\_order\_release}, \cppid{memory\_order\_acq\_rel}, \cppid{memory\_order\_seq\_cst}.
    \end{itemize}
\end{itemize}
\end{frame}

\begin{frame}[t,fragile]{\texttt{\textbf{atomic\_flag}}}
\begin{itemize}
  \item Tipo atómico \textmark{más simple posible}.
    \begin{itemize}
      \item \textenum{Dos estados posibles}: \textmark{activado} o \textmark{desactivado}.
      \item Siempre es libre de cerrojos.

      \mode<presentation>{\vfill\pause}
      \item Siempre hay que iniciarlos explícitamente a desactivado.
\begin{lstlisting}
std::atomic_flag f1 = ATOMIC_FLAG_INIT;
\end{lstlisting}

      \mode<presentation>{\vfill\pause}
      \item \textenum{Operaciones}:
        \begin{itemize}
          \item \textmark{Desactivar}: 
\begin{lstlisting}
f1.clear();
\end{lstlisting}
          \item \textmark{Activar y comprobar} valor previo: 
\begin{lstlisting}
f1.test_and_set();
\end{lstlisting}
        \end{itemize}
      \item Pueden indicar el orden de memoria de la operación. 
    \end{itemize}
\end{itemize}
\end{frame}

\begin{frame}[t,fragile]{Ejemplo: Un \emph{spin lock}}
\begin{itemize}
  \item Cerrojo que no hace uso de servicios del SO.
    \begin{itemize}
      \item Útil si los bloqueos van a durar muy poco tiempo y se quiere evitar problemas de cambio de contexto.
    \end{itemize}
\end{itemize}
\begin{block}{spin lock mutex}
\begin{lstlisting}
class spinlock_mutex {
private:
  std::atomic_flag f;
public:
  spinlock_mutex() : f{ATOMIC_FLAG_INIT} {}

  void lock() {
    while (f.test_and_set()) {}
  }
  void unlock() {
    flag.clear();
  }
};
\end{lstlisting}
\end{block}
\end{frame}

\begin{frame}[t,fragile]{\texttt{\textbf{atomic\_bool}}}
\begin{itemize}
  \item Más operaciones que \cppid{atomic\_flag}.
    \item Se puede iniciar y asignar con \cppid{bool}s.
    \item No se puede copiar de otro \cppid{atomic<bool>}.
    \item Modificación: \cppid{a.store(orden)}
    \item Consulta: \cppid{a.exchange(b, orden)}
    \item Conversión automática a \cppid{bool} (consistencia sec.): \cppid{a.load(orden)}
\end{itemize}
\begin{block}{Ejemplo}
\begin{lstlisting}
std::atomic<bool> a;
bool x = a.load(std::memory_order_acquire);
a.store(true);
x = a.exchange(false, std::memory_order_acq_rel);
\end{lstlisting}
\end{block}
\end{frame}

\begin{frame}[t,fragile]{Comparación e intercambio}
\begin{itemize}
  \item Compara el valor del atómico con un valor \textmark{esperado}.
    \begin{itemize}
      \item Si son iguales almacena el valor \textmark{deseado} en el atómico.
      \item Si no son iguales no modifica el atómico.
      \item Siempre retorna indicación de éxito o fracaso.
    \end{itemize}

  \mode<presentation>{\vfill\pause}
  \item Dos versiones:
    \begin{enumerate}
      \item \cppid{a.compare\_exchange\_weak(e,d)}:
        \begin{itemize}
          \item Permite fallos espúreos (cambio de contexto) en algunas arquitecturas.
          \item Puede comportarse como si \cppid{*}\cppkey{this}\cppid{!=e} incluso aunque sean iguales
        \end{itemize}
      \item \cppid{a.compare\_exchange\_strong(e,d)}:
        \begin{itemize}
          \item No permite fallos espúreos.
        \end{itemize}
    \end{enumerate}
%  \item Consecuencia:
%    \begin{itemize}
%      \item Cuando se usa en un bucle, la versión \emph{weak} da mejor rendimiento en algunas plataformas.
%      \item Cuando \emph{weak} necesita un bucle pero \emph{strong} no, \emph{strong} es mejor.
%    \end{itemize}

\end{itemize}
\end{frame}

\begin{frame}[t,fragile]{\texttt{\textbf{atomic\_address}}}
\begin{itemize}
  \item Acceso atómico a una dirección de memoria.
  \item No se puede copiar.
  \item Se puede copiar un puntero (\cppkey{void*}).
  \item Interfaz similar a atomic<bool>:
    \begin{itemize}
      \item \cppid{is\_lock\_free()}, \cppid{load()}, \cppid{store()}, \cppid{exchange()},
            \cppid{compare\_exchange\_weak()}, \cppid{compare\_exchange\_strong()}.
    \end{itemize}
  \item Operaciones adicionales.
    \begin{itemize}
      \item \cppid{fetch\_add()}, \cppid{fetch\_sub()}.
        \begin{itemize}
          \item Permiten especificar ordenamiento.
          \item Devuelven valor previo al cambio.
        \end{itemize}
    \end{itemize}
      \item \cppid{+=}, \cppid{-=}.
        \begin{itemize}
          \item Devuelven valor posterior al cambio.
          \item Todas usan aritmética de byte.
        \end{itemize}
      \item Otras aritméticas con \cppid{atomic<T*>}.
\end{itemize}
\end{frame}

\begin{frame}[t,fragile]{\texttt{\textbf{atomic<integral>}}}
\begin{itemize}
  \item Aplicable a todos los tipos integrales.
  \item Operaciones generales:
    \begin{itemize}
      \item \cppid{is\_lock\_free()}, \cppid{load()}, \cppid{store()}, \cppid{exchange()}, 
            \cppid{compare\_exchange\_weak()}, \cppid{compare\_exchange\_strong()}.
    \end{itemize}
  \item Operaciones aritméticas:
    \begin{itemize}
      \item \cppid{fetch\_add()}, \cppid{fetch\_sub()}, \cppid{fetch\_and()}, \cppid{fetch\_or()}, \cppid{fetch\_xor()}.
      \item \cppid{+=}, \cppid{-=}, \cppid{\&=}, \cppid{|=}, \cppid{\^{}=}.
      \item \cppid{++x}, \cppid{x++}, \cppid{--x}, \cppid{x--}
      \item No hay otras operaciones aritméticas (\cppid{*}, \cppid{/}, \cppid{\%}).
    \end{itemize}
\end{itemize}
\end{frame}

