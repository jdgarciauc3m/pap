\section{Modelos de consistencia}

\begin{frame}[t]{Consistencia secuencial}
\begin{itemize}
  \item \cppid{memory\_order\_seq\_cst}.
  \item El programa es consistente con una \textmark{vista secuencial}.
  \item Si todas las operaciones sobre atómicos son \textmark{secuencialmente consistentes}, 
        el comportamiento del programa multihilo es como si todas las operaciones 
        se realizasen en algún orden particular en un único hilo.
  \item No puede haber reordenaciones.
  \item Es el modelo más simple de razonar.
  \item Es el modelo más costoso en rendimiento.
\end{itemize}
\end{frame}

\begin{frame}[fragile]
\begin{columns}

\begin{column}{.6\textwidth}
\begin{block}{Acceso}
\begin{lstlisting}
td::atomic<bool> x, y;
std::atomic<int> z;

void f() {
  x.store(true, std::memory_order_seq_cst);
}

void g() {
  y.store(true, std::memory_order_seq_cst);
}

void h() {
  while (!x.load(std::memory_order_seq_cst)) {}
  if (y.load(std::memory_order_seq_cst))  ++ z;
}

void i() {
  while (!y.load(std::memory_order_seq_cst)) {}
  if (x.load(std::memory_order_seq_cst)) ++z;
}
\end{lstlisting}
\end{block}
\end{column}

\pause

\begin{column}{.4\textwidth}
\begin{block}{Lanzamiento de hilos}
\begin{lstlisting}
int main() {
  x = false;
  y = false;
  z = 0;

  std::thread t1{f};
  std::thread t2{g};
  std::thread t3{h};
  std::thread t4{i};

  t1.join();
  t2.join();
  t3.join();
  t4.join();

  assert(z.load() !=0);

  return 0;
}
\end{lstlisting}
\end{block}
\end{column}

\end{columns}
\end{frame}

\begin{frame}[t]{Consistencia sequencial: Análisis}
\input{12-atomic/seqconsist-ej2a.tkz}
\end{frame}

\begin{frame}[t]{Consistencia sequencial: Análisis}
\input{12-atomic/seqconsist-ej2b.tkz}
\end{frame}

\mode<article>{
\begin{itemize}
  \item Si \cppid{y.store()} en \cppid{g()} no \emph{sincroniza-con} \cppid{y.load()} en \cppid{h()}:
    \begin{itemize}
      \item La operación de \cppid{load} \emph{ocurre-antes} que la operación \cppid{store()}.
      \item Necesariamente \cppid{x.store()} de \cppid{f()} \emph{ocurre-antes} que \cppid{x.load()} en \cppid{i()}.
      \item El hilo \cppid{i()} incrementa la variable \cppid{z}.
    \end{itemize}
  \item Si \cppid{x.store()} en \cppid{f()} no \emph{sincroniza-con} \cppid{x.load()} en \cppid{i()}:
    \begin{itemize}
      \item La operación de \cppid{load()} \emph{ocurre-antes} que la operación \cppid{store()}.
      \item Necesariamente \cppid{y.store()} de \cppid{g()} \emph{ocurre-antes} que \cppid{y.load()} en \cppid{h()}.
      \item El hilo \cppid{h()} incrementa la variable \cppid{z}.
    \end{itemize}
\end{itemize}
}

\begin{frame}[t,fragile]{Ordenes secuencialmente no consistentes}
\begin{itemize}
\item Deja de haber un \textmark{orden global de los eventos}.
  \begin{itemize}
    \item Cada hilo puede tener \textbad{una vista diferente}.
      \begin{itemize}
        \item Los hilos pueden no estar de acuerdo en el orden de los eventos.
      \end{itemize}

    \pause
    \item Pero, \ldots
      \begin{itemize}
        \item \textgood{Todos los hilos deben estar de acuerdo en el orden de modificación de cada variable}.
      \end{itemize}
  \end{itemize}
  \mode<presentation>{\vfill\pause}
\item \textenum{Alternativas}:
  \begin{itemize}
    \item Ordenamiento \textmark{relajado}.
    \item Ordenamiento \textmark{adquisición liberación}.
  \end{itemize}
\end{itemize}
\end{frame}

\begin{frame}[t]{Ordenamiento relajado}
\begin{itemize}
\item \cppid{memory\_order\_relaxed}
  \item Operaciones relajadas sobre atómicos \textmark{no participan} en la relación \textgood{sincroniza-con}.

  \mode<presentation>{\vfill\pause}
  \item Operaciones sobre misma variable en mismo hilo \textmark{si cumplen} relación \textgood{ocurre-antes}.
    \begin{itemize}
      \mode<presentation>{\vfill}
      \item \textbad{No se pueden reordenar} accesos a una variable atómica dentro de un mismo hilo.
      \mode<presentation>{\vfill}
      \item Una vez que un hilo ha visto un valor de una variable \textbad{no pude ver} un valor más antiguo de esa variable.
    \end{itemize}
\end{itemize}
\end{frame}

\begin{frame}[fragile]{Ejemplo}
\begin{columns}

\begin{column}{.6\textwidth}
\begin{block}{Acceso a datos}
\begin{lstlisting}
std::atomic<bool> x, y; std::atomic<int> z;

void f() {
  x.store(true, std::memory_order_relaxed);
  y.store(true, std::memory_order_relaxed);
}
void g() {
  while (!y.load(std::memory_order_relaxed)) {}
  if (x.load(std::memory_order_relaxed)) { ++z; }
}

int main() {
  x=false; y=false; z=0;
  std::thread t1{f}; std::thread t2{g};
  t1.join(); t2.join();
  return 0;
}

\end{lstlisting}
\end{block}
\end{column}

\begin{column}{.4\textwidth}
\input{12-atomic/relconsist-ej.tkz}
\end{column}

\end{columns}
\end{frame}


\mode<article>{
\begin{itemize}
  \item Observaciones:
    \begin{itemize}
      \item Todas las operaciones sobre \cppid{x} e \cppid{y} son relajadas.
      \item El incremento de \cppid{z} se realiza con consistencia secuencial.
    \end{itemize}
  \vfill
  \item Consecuencias:
    \begin{itemize}
      \item No es posible establecer ninguna relación \emph{sincroniza-con} entre los hilos
            a través de \cppid{x} o \cppid{y}.
      \item Los cambios sobre \cppid{x} e \cppid{y} pueden verse en cualquier orden en \cppid{g()}.
      \item La variable \cppid{z} podría incrementarse o no.
    \end{itemize}
\end{itemize}
}

\begin{frame}[t]{Orden de adquisición/liberación}
\begin{itemize}
  \item \cppid{memory\_order\_acquire}, \cppid{memory\_order\_release}, \cppid{memory\_order\_acq\_rel}.
  
  \mode<presentation>{\vfill}
  \item Nivel \textmark{intermedio} de sincronización.
  
  \mode<presentation>{\vfill\pause}
  \item Una operación de \textgood{liberación} que \textmark{escribe un valor} 
        se \textgood{sincroniza-con} una operación de \textgood{adquisición} que \textmark{lee dicho valor}.
  
  \mode<presentation>{\vfill}
  \item \textenum{Impacto}:
    \begin{itemize}
      \item \textbad{Distintos hilos pueden ver distintos órdenes}.
      \item \textbad{No todos los órdenes son posibles}.
    \end{itemize}
\end{itemize}
\end{frame}


\begin{frame}[t,fragile]
\begin{columns}

\begin{column}{.6\textwidth}
\begin{block}{Acceso}
\begin{lstlisting}
std::atomic<bool> x, y;
std::atomic<int> z;

void f() {
  x.store(true, std::memory_order_release);
}

void g() {
  y.store(true, std::memory_order_release);
}

void h() {
  while (!x.load(std::memory_order_acquire)) {}
  if (y.load(std::memory_order_acquire))  ++ z;
}

void i() {
  while (!y.load(std::memory_order_acquire)) {}
  if (x.load(std::memory_order_acquire)) ++z;
}
\end{lstlisting}
\end{block}
\end{column}

\begin{column}{.4\textwidth}
\begin{block}{Lanzamiento de hilos}
\begin{lstlisting}
int main() {
  x = false;
  y = false;
  z = 0;

  std::thread t1{f};
  std::thread t2{g};
  std::thread t3{h};
  std::thread t4{i};

  t1.join();
  t2.join();
  t3.join();
  t4.join();

  assert(z.load() !=0);
  return 0;
}
\end{lstlisting}
\end{block}
\end{column}

\end{columns}
\end{frame}

\begin{frame}[t]{Análisis}
\input{12-atomic/relacq-consist-ej.tkz}

\mode<presentation>{\vfill\pause}
\begin{itemize}
  \item Múltiples ordenes son posibles porque no hay relaciones
        \emph{acquire} $\rightarrow$ \emph{release}.
\end{itemize}
\end{frame}

\mode<article>{
\begin{itemize}
  \item Observaciones:
    \begin{itemize}
      \item La operación \cppid{x.store()} de \cppid{f()} \emph{sincroniza-con} 
            la operación \cppid{x.load()} de \cppid{h}.
        \begin{itemize}
          \item Release/Acquire.
        \end{itemize}
      \item La operación \cppid{y.store()} de \cppid {g()} \emph{sincroniza-con}
            la operación \cppid{y.load()} de \cppid{i()}.
        \begin{itemize}
          \item Release/Acquire.
        \end{itemize}
    \end{itemize}
  \vfill
  \item Consecuencias:
    \begin{itemize}
      \item La operación \cppid{y.load()} de \cppid{h()} podría dar como resultado \cppkey{true} o \cppkey{false}.
      \item La operación \cppid{x.load()} cd cppid{i()}podría dar como resultado \cppkey{true} o \cppkey{false}.
    \end{itemize}
\end{itemize}
}

\begin{frame}[fragile]{Combinación de órdenes}
\begin{itemize}
\item Se puede obtener un efecto \textmark{equivalente}
      a \textgood{consistencia secuencial} con \textmark{menos coste}.
\end{itemize}
\begin{columns}

\begin{column}{.5\textwidth}
\begin{block}{Acceso}
\begin{lstlisting}[basicstyle=\tiny]
std::atomic<bool> x, y; std::atomic<int> z;

void f() {
  x.store(true, std::memory_order_relaxed);
  y.store(true, std::memory_order_release);
}
void g() {
  while (!y.load(std::memory_order_acquire)) {}
  if (x.load(std::memory_order_relaxed)) ++z;
}
int main() {
  x = false; y = false; z = 0;
  std::thread t1{f}; std::thread t2{g};
  t1.join(); t2.join();
  assert(z.load() !=0);

  return 0;
}
\end{lstlisting}
\end{block}
\end{column}

\begin{column}{.5\textwidth}
\input{12-atomic/comb-consist-ej.tkz}
\end{column}

\end{columns}
\end{frame}

