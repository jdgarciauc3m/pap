\section{Modelo de memoria}

\begin{frame}{C++ y consistencia de memoria}
\begin{itemize}
  \item C++11 define un \textmark{modelo de concurrencia} propio como parte del propio lenguaje.

  \mode<presentation>{\vfill\pause}
  \item \textgood{Objetivo}: 
        Evitar la necesidad de escribir código en lenguajes de más bajo nivel (C, ensamblador, ...) para obtener mayores prestaciones.
    \begin{itemize}
      \item Tipos atómicos.
      \item Mecanismos de sincronización de bajo nivel.
    \end{itemize}

  \mode<presentation>{\vfill\pause}
  \item Permite la construcción de \textmark{estructuras de datos libres de cerrojos}.
\end{itemize}
\end{frame}

\begin{frame}{Objetos y posiciones de memoria}
\begin{itemize}
  \item \textgood{Objeto}: Es una región de almacenamiento.
    \begin{itemize}
      \item Una secuencia de uno o varios bytes.
    \end{itemize}

  \mode<presentation>{\vfill\pause}
  \item \textgood{Posición de memoria}: Es un objeto de un tipo escalar o una secuencia de campos de bits adyacentes.

  \mode<presentation>{\vfill\pause}
  \item \textmark{Un objeto se almacena en una o varias posiciones de memoria}.
\end{itemize}
\end{frame}

\begin{frame}[fragile]{Ejemplo}
\begin{itemize}
  \item Estructura:
\begin{lstlisting}
struct {
  int i;
  char c;
  int d: 10;
  int e: 16;
  double f;
};
\end{lstlisting}
  \item Posiciones de memoria:
    \begin{enumerate}
      \item \cppid{i}.
      \item \cppid{c}.
      \item \cppid{d}, \cppid{e}.
      \item \cppid{f}.
    \end{enumerate}
\end{itemize}
\end{frame}

\begin{frame}{Reglas}
\begin{itemize}
  \item Dos hilos pueden acceder a \textmark{posiciones de memoria distintas} 
        de forma simultánea.

  \mode<presentation>{\vfill\pause}
  \item Dos hilos pueden acceder a la \textmark{misma posición de memoria} 
        de forma simultánea si ambos accesos son de \textmark{lectura}.

  \mode<presentation>{\vfill\pause}
  \item Si dos hilos intentan acceder de forma simultánea a la \textmark{misma posición de memoria} 
        y alguno de los accesos es de \textmark{escritura} existe una 
        \textmark{condición de carrera potencial}.
    \begin{itemize}
      \item Depende de si se obliga un \textmark{orden entre ambos accesos}.
    \end{itemize}
\end{itemize}
\end{frame}

\begin{frame}{Ordenamiento y condiciones de carrera}
\begin{itemize}
  \item \textenum{Solución clásica}: Uso de mecanismos de \textmark{sincronización}.
    \begin{itemize}
      \item Permite garantizar la \textmark{exclusión mutua}.
      \item Basado en SO $\rightarrow$ Puede ser costoso.
    \end{itemize}

  \mode<presentation>{\vfill\pause}
  \item \textbf{Alternativa}: Uso de \textmark{operaciones atómicas} para garantizar 
        \textmark{ordenamiento}.
 
  \mode<presentation>{\vfill\pause}
    \begin{itemize}
      \item Si no se establece el \textmark{orden entre dos accesos} a una posición de memoria.
      \item alguno de los accesos \textmark{no es atómico}, 
      \item y al menos uno de los accesos es una \textmark{escritura}, 
      \item estos constituyen una \textbad{carrera de datos} y el 
        \textbad{comportamiento del programa no está definido}.
    \end{itemize}
\end{itemize}
\end{frame}

\begin{frame}{Orden de modificación}
\begin{itemize}
  \item \textgood{Orden de modificación}: Secuencia de escrituras sobre un objeto.
    \begin{itemize}
      \item Si dos hilos ven distintos ordenes de modificación sobre un objeto hay una \textbad{carrera de datos}.
      \item Las modificaciones no tienen por qué ser visibles en el mismo instante en todos los hilos.
    \end{itemize}

  \mode<presentation>{\vfill\pause}
  \item Una lectura posterior a una escritura en un mismo hilo observa 
        el valor escrito o un valor posterior en su \textmark{orden de modificación}.
\end{itemize}
\end{frame}

