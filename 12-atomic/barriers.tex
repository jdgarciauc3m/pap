\section{Barreras}

\begin{frame}[fragile]{Barreras}
\begin{itemize}
\item \textmark{Fuerzan} \textgood{ordenación} sin modificar datos.
\end{itemize}

\begin{columns}

\begin{column}{.7\textwidth}
\begin{block}{Ejemplo}
\begin{lstlisting}
std::atomic<bool> x, y;
std::atomic<int> z;

void f() {
  x.store(true, std::memory_order_relaxed);
  std::atomic_thread_fence(std::memory_order_release);
  y.store(true, std::memory_order_relaxed);
}

void g() {
  while (!y.load(std::memory_order_relaxed)) {}
  std::atomic_thread_fence(std::memory_order_acquire);
  if (x.load(std::memory_order_relaxed)) ++z;
}
\end{lstlisting}
\end{block}
\end{column}

\begin{column}{.3\textwidth}
\begin{block}{Hilos}
\begin{lstlisting}
int main() {
  x = false;
  y = false;
  z = 0;

  std::thread t1(f);
  std::thread t2(g);

  t1.join();
  t2.join();

  assert(z.load() !=0);
  return 0;
}
\end{lstlisting}
\end{block}
\end{column}

\end{columns}
\end{frame}

\begin{frame}{Barreras: Análisis}
\input{12-atomic/barriers-ej.tkz}
\end{frame}

\mode<article>{
\begin{itemize}
  \item Observaciones:
    \begin{itemize}
      \item Una operación relajada seguida de una barrera de liberación, tiene
            el mismo efecto que una operación con semántica de liberación.
      \item Una operación relajada despuś una barrera de adquisición, tiene
            el mismo efecto que una operación con semántica de adquisición.
      \item La operación \cppid{y.load()} de \cppid{g()} lee el valor almacenado por
            la operación \cppid{y.store()} de \cppid{f()}.
        \begin{itemize}
          \item Existe una relación \emph{sincroniza-con} de la barrera de \cppid{f()}
                a la barrera de \cppid{g()}.
        \end{itemize}
    \end{itemize}
  \item Consecuencias:
    \begin{itemize}
      \item Existe una relación \emph{ocurre-antes} de la operación \cppid{x.store()} de \cppid{f()}
            a la operación \cppid{x.load()} de \cppid{g()}.
      \item Siempre se incrementa \cppid{z}.
    \end{itemize}
\end{itemize}
}

