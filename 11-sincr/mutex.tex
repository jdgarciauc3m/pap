\section{Objetos mutex}

\subsection{Tipos de mutex}

\begin{frame}{Clasificación de \emph{mutex}}
\begin{itemize}
  \item Representan el acceso exclusivo a un recurso.
    \begin{itemize}
      \item \cppid{mutex}: \emph{Mutex} básico no recursivo.
      \item \cppid{recursive\_mutex}: Un \emph{mutex} que puede ser adquirido más de una vez por un mismo hilo.
      \item \cppid{timed\_mutex}: \emph{Mutex} no recursivo con operaciones con tiempo límite.
      \item \cppid{recursive\_timed\_mutex}: \emph{Mutex} recursivo con operaciones con tiempo límite.
    \end{itemize}
  \item Solamente un hilo puede poseer un \emph{mutex} en un momento dado.
    \begin{itemize}
      \item Adquirir un \emph{mutex} $\rightarrow$ obtener acceso exclusivo al objeto.
        \begin{itemize}
          \item Operación bloqueante.
        \end{itemize}
      \item Liberar un \emph{mutex} $\rightarrow$ Liberar el acceso exclusivo al objeto.
        \begin{itemize}
          \item Permite que otro hilo obtenga el acceso.
        \end{itemize}
    \end{itemize}
\end{itemize}
\end{frame}

\subsection{Mutex básico}

\begin{frame}{Operaciones}
\begin{itemize}
  \item Construcción y destrucción:
    \begin{itemize}
      \item Se puede construir por defecto.
      \item No se pueden copiar ni mover.
      \item El destructor puede dar comportamiento no definido si el \emph{mutex} no está libre.
    \end{itemize}
  \item Adquisición y liberación:
    \begin{itemize}
      \item \cppid{m.lock()}: Adquiere el \emph{mutex} de forma bloqueante.
      \item \cppid{m.unlock()}: Libera el \emph{mutex}.
      \item \cppid{r = m.try\_lock()}: Intenta adquirir el \emph{mutex}, devolviendo indicación de éxito.
    \end{itemize}
  \item Otras:
    \begin{itemize}
      \item \cppid{h = m.native\_handle()}: Devuelve el identificador dependiente de la plataforma de tipo
            \cppid{native\_handle\_type}.
    \end{itemize}
\end{itemize}
\end{frame}

\begin{frame}[fragile]{Ejemplo}
\begin{lstlisting}
mutex mutex_salida;

void imprime(int x) {
  mutex_salida.lock();
  cout << x << endl;
  mutex_salida.unlock();
}

void imprime(double x) {
  mutex_salida.lock();
  cout << x << endl;
  mutex_salida.unlock();
}

void f() {
  thread t1{imprime, 10};
  thread t2(imprime, 5.5};
  thread t3(imprime, 3);
 
  t1.join(); t2.join(); t3.join();
}
\end{lstlisting}
\end{frame}

\begin{frame}[fragile]{Exclusión mutua recursiva}
\begin{lstlisting}
mutex mutex_salida;

template <typename T, typename ... U>
void imprime(T x, U resto ...) {
  mutex_salida.lock();
  cout << x;
  imprime(resto...); // Deadlock
  mutex_salida.unlock();
}

recursive_mutex rmutex_salida;

template <typename T, typename ... U>
void imprime(T x, U resto ...) {
  rmutex_salida.lock();
  cout << x;
  imprime(resto...); // No hay deadlock
  rmutex_salida.unlock();
}
\end{lstlisting}
\end{frame}

\subsection{Exclusión mutua y errores}

\begin{frame}[fragile]{Errores en exclusión mutua}
\begin{itemize}
  \item En caso de error se lanza la excepción \cppid{system\_error}.
  \item Códigos de error:
    \begin{itemize}
      \item \cppid{resource\_deadlock\_would\_occur}.
      \item \cppid{resource\_unavailabe\_try\_again}.
      \item \cppid{operation\_not\_permitted}.
      \item \cppid{device\_or\_resource\_busy}.
      \item \cppid{invalid\_argument}.
    \end{itemize}
\begin{lstlisting}
mutex m;
try {
  m.lock();
  // 
  m.lock();
}
catch (system_error & e) {
  cerr << e.what() << endl;
  cerr << e.code() << endl;
}
\end{lstlisting}
\end{itemize}
\end{frame}

\subsection{Operaciones con tiempo límite}

\begin{frame}{Tiempo límite}
\begin{itemize}
  \item Operaciones soportadas por \cppid{timed\_mutex} y \cppid{recursive\_timed\_mutex}.
  \item Añaden operaciones de adquisión con especificación de tiempo límite.
    \begin{itemize}
      \item \cppid{r = m.try\_lock\_for(d)}: Intenta adquirir el \emph{mutex} durante una duración
            \cppid{d}, devolviendo indicación de éxito.
      \item \cppid{r = m.try\_lock\_until(t)}: Intenta adquirir el \emph{mutex} hasta un momento
            en el tiempo, devolviendo indicación de éxito.
    \end{itemize}
\end{itemize}
\end{frame}
