\section{Paralelismo y altas prestaciones}

\begin{frame}[t]{¿Por qué TLP?}
\begin{itemize}
  \item Algunas aplicaciones presentan más \textgood{paralelismo natural} que 
        el que se puede conseguir con \textmark{arquitecturas segmentadas}.
    \begin{itemize}
      \item Servidores, aplicaciones científicas, \ldots
    \end{itemize}

  \mode<presentation>{\vfill\pause}
  \item Emergen \textenum{dos modelos}:
    \begin{itemize}
      \item \textgood{Paralelismo a nivel de hilos (TLP)}:
        \begin{itemize}
          \item \textmark{Hilo}: Proceso con sus propias instrucciones y datos.
          \item Puede ser parte un programa o programa independiente.
          \item Cada hilo tiene asociado su \textmark{estado} 
                (instrucciones, datos, PC, registros, \ldots).
        \end{itemize}
      \item \textgood{Paralelismo a nivel de datos (DLP)}:
        \begin{itemize}
          \item Operaciones idénticas sobre distintos datos.
        \end{itemize}
    \end{itemize}
\end{itemize}
\end{frame}

\begin{frame}[t]{TLP}
\begin{itemize}
  \item \textgood{ILP} explota paralelismo implícito dentro un bloque básico
        o en bucles.

  \mode<presentation>{\vfill}
  \item \textgood{TLP} utiliza múltiples hilos de ejecución que son 
        inherentemente paralelos.

  \mode<presentation>{\vfill\pause}
  \item \textenum{Objetivo de TLP}:
    \begin{itemize}
      \item Utilizar múltiples flujos de instrucciones para mejorar:
        \begin{itemize}
          \item \textmark{Tasa de procesamiento} de computadores que ejecutan muchos programas.
          \item \textmark{Tiempo de ejecución} de programas multi-hilo.
        \end{itemize}
    \end{itemize}
\end{itemize}
\end{frame}

\begin{frame}[t]{Ejecución multi-hilo}
\begin{itemize}
  \item Múltiples hilos comparten las unidades funcionales de un procesador solapando su uso.
    \begin{itemize}
      \item Necesidad de replicar n-veces el estado.
        \begin{itemize}
          \item Banco de registros, PC, tabla de páginas (si hilos no pertenecen al mismo programa).
          \item Memoria compartida mediante mecanismos de memoria virtual.
          \item Hardware para cambio de hilo rápido.
        \end{itemize}

      \mode<presentation>{\vfill\pause}
      \item \textenum{Tipos}:
        \begin{itemize}
          \item \textmark{Grano fino}: Cambio de hilo en cada instrucción.
          \item \textmark{Grano grueso}: Cambio de hilo en detenciones (ej. Fallo de caché).
          \item \textmark{Multi-hilo simultáneo}: Grano fino con emisión múltiple y planificación dinámica.
        \end{itemize}
    \end{itemize}
\end{itemize}
\end{frame}

\begin{frame}[t]{Multi-hilo de grano fino}
\begin{itemize}
  \item Se alterna entre hilos en cada instrucción.
    \begin{itemize}
      \item Se entrelaza la ejecución de los hilos.
      \item Normalmente se hace \emph{round-robin}.
      \item Se excluyen de \emph{round-robin} hilos en detención (\emph{stall}).
      \item El procesador debe poder cambiar de hilo en cada ciclo de reloj.
    \end{itemize}

  \mode<presentation>{\vfill\pause}
  \item \textgood{Ventaja}:
    \begin{itemize}
      \item Puede ocultar detenciones cortas y largas.
    \end{itemize}

  \mode<presentation>{\vfill}
  \item \textbad{Desventaja}:
    \begin{itemize}
      \item Retrasa la ejecución de hilos individuales por reparto.
    \end{itemize}

  \mode<presentation>{\vfill}
  \item \textmark{Ejemplo}: Sun Niagara.
\end{itemize}
\end{frame}

\begin{frame}[t]{Multi-hilo de grano grueso}
\begin{itemize}
  \item Cambia de hilo solamente en detenciones largas.
    \begin{itemize}
      \item \textmark{Ejemplo}: Fallo en caché L2.
    \end{itemize}

  \mode<presentation>{\vfill\pause}
  \item \textgood{Ventajas}:
    \begin{itemize}
      \item No hace falta cambio de hilo excesivamente rápido.
      \item No retrasa hilos individuales.
    \end{itemize}

  \mode<presentation>{\vfill}
  \item \textbad{Desventajas}:
    \begin{itemize}
      \item Se debe vaciar o congelar el pipeline.
      \item Se debe llenar el pipeline con instrucciones del nuevo hilo (latencia).
    \end{itemize}

  \mode<presentation>{\vfill}
  \item Apropiado cuando rellenado del pipeline tarda mucho menos que tiempo de detención.
    \begin{itemize}
      \item \textmark{Ejemplo}: IBM AS/400.
    \end{itemize}
\end{itemize}
\end{frame}

\begin{frame}[t]{SMT: Multi-hilo simultáneo}
\begin{itemize}
  \item \textgood{Idea}: Procesadores con planificación dinámica 
        ya tienen muchos mecanismo de soporte para multi-hilo.
    \begin{itemize}
      \item Grandes conjuntos de registros virtuales.
        \begin{itemize}
          \item Registros para múltiples hilos.
        \end{itemize}
      \item Renombrado de registros.
        \begin{itemize}
          \item Evita mezcla en acceso a registros de hilos.
        \end{itemize}
      \item Finalización fuera de orden.
    \end{itemize}

  \mode<presentation>{\vfill\pause}
  \item \textenum{Modificaciones}:
    \begin{itemize}
      \item Tabla de renombrado por hilo.
      \item Registros PC separados.
      \item ROB separados.
    \end{itemize}

  \mode<presentation>{\vfill}
  \item \textmark{Ejemplos}: Intel Core i7, IBM Power 7

\end{itemize}
\end{frame}

\begin{frame}[t]{TLP: Resumen}
\makebox[\textwidth][c]{
\input{02-fundcap/tlp-alt.tkz}
}
\end{frame}
